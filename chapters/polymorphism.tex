\chapter{Polymorphism}\label{ch:polymorphism}

In order to generalize functions over data types there have been several proposals to abstract over types in different programming languages:
These can be summarised in three categories: parametric polymorphism, subtyping and ad-hoc polymorphism.

\section{Parametric polymorphism}\label{sec:parmetric-polymorphism}

We can define functions without knowing the concrete representation of the arguments and result types (e.g. \texttt{id : a -> a}) % cite something
Abstract algorithms detached from concrete type representation.
Allows for 'theorems for free'. \cite{wadlertheorems}

\section{Subtyping}\label{sec:subtyping}

In many cases the specific semantics of types exhibit a hierarchy.
In Object-oriented programming this hierarchy is given in the form of sub- and superclasses.
We can express the relationship between super- and subclasses, or more generally super- and subtypes,
in the form that all properties of the superclass is also exhibited in the subclass. \cite{subtyping}

In the case of OO-languages this means that, if the class \texttt{SubC} is a subclass of \texttt{SuperC},
then any method defined in \texttt{SuperC} is also going to be defined for objects of \texttt{SubC}.
This enables us to use an object \texttt{SubC} wherever a \texttt{SuperC} is expected.

We denote $T \leq S$ for $T$ is a subtype of $S$.
Syntactically, this implies that if we have obatined the judgement $e : T$, we also have $e : S$.
Therefore, we can use $e$ in any context that expects the usage of a term of type $S$.
Semantically, the subtyping relation can be understood analogously to sets in terms of the subset relationship $\subseteq$,
meaning all terms $e$ of type $T$ are also of type $S$.
\cite{reynolds_1998}

This permits many useful features in programming languages such as the reuse and abstraction of code to the supertypes and implicit coercions from a subtype to a supertype.


\section{Ad-hoc polymorphism}\label{sec:ad-hoc-polymorphism}
% overloading vs type classes
% paper by wadler

In some - mostly imperative - languages it is possible to simply overload functions (e.g. we may define \texttt{+} both on integers and on float values, the correct implementation is then picked based on the argument type).
However, there is usually no way to express this in the type system of these languages. % cite...

If our type system doesn't allow for ad-hoc polymorphism it may seem neccessary to write verbose code for basic function with respect to every concrete type it should be used for.
An intuitive example for this (that is also the motivation for type classes in the original proposal) are arithmetic operators.
We simply cannot define \texttt{(+) : Int -> Int -> Int} and then also \texttt{(+) : Float -> Float -> Float} in a different implementation, on which both types definetly rely on under the hood.
But these types have nonetheless something in common. Namely that they both stand for \emph{numerical} values, that hence support the usual arithmetic operations, like addition, multiplication, division and so on.

The idea of type classes is to generalise attributes of types with appropriate function.
The class \mintinline{Haskell}|Num| in Haskell expects that we can implement a number of numerical functions for a type $\tau$ if it is ought to be a member of the \mintinline{Haskell}|Num| type class.

Shortened definition of the \mintinline{Haskell}|Num| type class in Haskell.
\footnote{As can be found in the default \texttt{Prelude}: \url{https://hackage.haskell.org/package/base-4.16.2.0/docs/Prelude.html\#t:Num}}

\begin{minted}{Haskell}
class  Num a  where
    (+), (-), (*) :: a -> a -> a
\end{minted}

An instance would look like:

\begin{minted}{Haskell}
instance  Num Int  where
    (+) = intAdd
    (-) = intMinus
    (*) = intMul
\end{minted}





% later:
% category theory: functors, monoids, monads, etc.

In the end, this enables us to use elaborate concepts such as functors and monads to reason about programs.

\cite{wadlerblott}

\section{The problem of type class coherence}\label{sec:coherence}
% for each type there t may be globally at most one instance C t defined
% discuss overlapping instances
% can already be tricky in a modular system
% more challenging with subtyping

Even though the general concept of type classes introduces a general meaning for each type class.
The evaluation still strongly depends on implementation details found in specific instances.
For example, for the \texttt{Ord} type class we may choose to implement the ordering in ascending or descending order.
It is therefore crucial, that for each type the corresponding instance - if it exists - is uniquely determined by the type.

Reynolds \cite{reynolds_coherence} describes the issue of coherence as follows:

\begin{quote}
    When a programming language has a sufficiently rich type structure, there can be more than one proof of the same
    typing judgment; potentially this can lead to semantic ambiguity since the semantics of a typed language is a function
    of such proofs. When no such ambiguity arises, we say that the language is coherent.
\end{quote}

For type classes, this means that no two instances should be able to be resolved for the same type.
A rather obvious example would be to define two different instances for the same type.
For example one instance of \mintinline{Haskell}{Ord Int}, one with ascending and one with descending order.
The ambiguity arises as soon as we make use of these instances and it is no longer clear which one should be picked for evaluation.

More surprisingly type class coherence is already violated for overlapping instances.
As part of the standard prelude we find both \mintinline{Haskell}{instance Show a => Show [a]} and \mintinline{Haskell}{instance Show String}
(with \mintinline{Haskell}{String} being a type synonym for \mintinline{Haskell}{[Char]}).
The \texttt{Show} instance for Strings differs from the more general instance for lists.

Ambiguous programs should generally not be typeable.
One example, also mentioned in the Haskell 98 report \cite{Haskell98} is this short program which simply reads a string to a data type and then converts it back to string without specifying which data type is being used:

\begin{minted}{Haskell}
    f :: String -> String
    f str = let x = read str in show x
\end{minted}

There may be multiple types that satisfy the type class constraints.
The specific implementation of \mintinline{Haskell}{show :: forall a. (Show a) => a -> String} and \mintinline{Haskell}{read :: forall a.(Read a) -> String -> a} is therefore unknown.

In the Haskell98 standard, type class coherence is guaranteed by the syntactical equivalence of resolved instances.
For each Haskell type there may be at most one instance defined for each type class.


\subsection{Example}

Consider superclasses in Haskell.
E.g. the type class \mintinline{Haskell}{Eq} is a superclass of \mintinline{Haskell}{Ord}, written \mintinline{Haskell}{Eq a => Ord a}.
This means that for each type that we want to define an ordering for already needs to have equality defined on.

For example, given an instance for \mintinline{Haskell}{Ord (Maybe a)} we can then derive that an instance for \mintinline{Haskell}{Eq a} has to exist.
As shown in the diagram, we can take different paths to do so but type class coherence guarantees that no matter which path we choose to resolve the instance, we will always find \emph{the same} instance.
In Haskell98 type class coherence guarantees that all such diagrams commute.
So even if there may be different ways to resolve type class constraints, all of them preserve the same semantics.

\begin{tikzcd}
    &  & \mintinline{Haskell}|Ord (Maybe a)| \arrow[llddd, "{\footnotesize\mintinline{Haskell}|class Eq a => Ord a|}"'] \arrow[rrddd, "{\footnotesize\mintinline{Haskell}|instance Ord a => Ord (Maybe a)|}"] &  &                            \\
    &  &                                &  &                            \\
    &  &                                &  &                            \\
    \mintinline{Haskell}|Eq (Maybe a)| \arrow[rrddd, "{\footnotesize\mintinline{Haskell}|instance Eq a => Eq (Maybe a)|}"'] &  &                                &  & \mintinline{Haskell}|Ord a| \arrow[llddd, "{\footnotesize \mintinline{Haskell}|class Eq a => Ord a|}"] \\
    &  &                                &  &                            \\
    &  &                                &  &                            \\
    &  & \mintinline{Haskell}|Eq a|                  &  &                           
\end{tikzcd}

Even though there are multiple ways to derive an instance for \mintinline{Haskell}|Eq a| from an instance of \texttt{Ord (Maybe a)}, the derived instance has to be uniquely determined.
Since the diagram commutes, there has to be exactly one instance for \mintinline{Haskell}|Eq a|.

Uniqueness or non-ambiguity of type classes.
