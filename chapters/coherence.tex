\chapter{Type Class Coherence}
\label{ch:coherence}

\section{Classes and Instances}

We introduce top-level declarations for type classes and type class instances.
Fig. \ref{fig:class-showable} shows an example class declaration for the covariant type class $\Showable$.
$\alpha$ is the bound type variable used in the type signature of the type class method.
The superscript $^+$ constraints the variable to only occur in input positions in the signature of the type class method $\mathit{show}$.
At the same time, this means that the type class is covariant which is important information for type class resolution.
The superscript $^-$, conversely, constraints the variable to only occur in output positions and is used for contravariant type classes.
If there is no such superscript, the type class is declared as invariant.

\begin{figure}[ht]
    \centering
    \begin{flalign*}
        & \class{\Showable}{\alpha^+} :            \\
        & \; \; \mathit{show} : \alpha \to \String
    \end{flalign*}
    \caption{Class declaration for type class $\Showable$}
    \label{fig:class-showable}
\end{figure}

Instance declarations are dependent on type class declarations.
An example instance declaration for $\Showable(\Nat)$ is given in fig. \ref{fig:instance-showable}.
We define a type for which the instance should be declared (in this case $\Nat$) and replace any occurences of the type variable in the class declaration with this type.
We then need to define an implementation of the declared type class method.

\begin{figure}[ht]
    \centering
    \begin{flalign*}
        & \instance{\Showable}{\Nat} :                                                         \\
        & \; \; \mathit{show} : \Nat \to \String                                               \\
        & \; \; \mathit{show} := \lambda x. \caseof{x}                                         \\
        & \; \; \; \; Z \Rightarrow \mathit{"Z"}                                               \\
        & \; \; \; \; S(n) \Rightarrow \mathit{append} \; \mathit{"S"} \; (\mathit{show} \; n)
    \end{flalign*}
    \caption{Instance declaration for $\Showable(\Nat)$}
    \label{fig:instance-showable}
\end{figure}

In our examples there is only one method declared per type class but in practise this can be an arbitrary number.
It is common practise in Haskell, for example, to only declare a minimal subset of type class methods and use a default implementation based on them for the other type class methods.

\section{Motivation}

Consider the following instance declarations (fig. \ref{fig:showable-nat}):


\begin{figure}[ht]
    \centering
    \begin{subfigure}{0.4\textwidth}
        \begin{flalign*}
             & \instance{\Showable}{\Nat} :                                                         \\
             & \; \; \mathit{show} : \Nat \to \String                                               \\
             & \; \; \mathit{show} := \lambda x. \caseof{x}                                         \\
             & \; \; \; \; Z \Rightarrow \mathit{"Z"}                                               \\
             & \; \; \; \; S(n) \Rightarrow \mathit{append} \; \mathit{"S"} \; (\mathit{show} \; n)
        \end{flalign*}
    \end{subfigure}
    \hfill
    \begin{subfigure}{0.4\textwidth}
        \begin{flalign*}
             & \instance{\Showable}{\Nat} :                                                         \\
             & \; \; \mathit{show} : \Nat \to \String                                               \\
             & \; \; \mathit{show} := \lambda x. \caseof{x}                                         \\
             & \; \; \; \; Z \Rightarrow \mathit{"0"}                                               \\
             & \; \; \; \; S(n) \Rightarrow \mathit{append} \; \mathit{"S"} \; (\mathit{show} \; n)
        \end{flalign*}
    \end{subfigure}
    \caption{Incoherent instance declarations for $\Showable(\Nat)$}
    \label{fig:showable-nat}
\end{figure}

If there is more than one instance declaration in the same environment for the same type class and type,
the program may be indeterministic.
If we call the method $\mathit{show}$ for an argument of type $\Nat$,
it is therefore undecidable which instance is being picked.

Incoherence causes unpredictable behavior of programs.
Not only does this potentially produce unexpected results, it also prevents us from reasoning about programs.
E.g. if the $\mathit{insert}$ operation for search trees relies on the type class $\mathit{Ord}$ for which we have incoherent instances,
the search tree properties may be violated. \cite{Kilpatrick2019-cy}

Moreover, in the presence of subtyping, instances for types related by the subtyping relation may not be declared for the same type class because they allow resolution for the same type class constraint from different instances.

Therefore, we want to define a criterion which globally guarantees that no such ambiguity can arise when resolving type class constraints.


\section{Definition}

Using witnesses we may formalize type class coherence:

\begin{definition}
  Type class resolution is coherent if and only if for any pair of witnesses $i,j : \Phi(\tau)$ that witness the same predicate application we have $i \opeq j$.
\end{definition}

In our  system, we can define operational equality $\opeq$ simply as two witnesses of a predicate application containing the same leaf, i.e. the same instance.
Therefore, the usage of these witnesses can always be expected to behave in the same way.
Witnesses always point to exactly one instance definition.
Thus, in our system it is sufficient show that for any pair of witnesses the leaf (i.e. the instance) is the same.
In other systems, that use coercions for subtypes, additional steps may have to be taken.

\section{Consequences}

Following this there are some constraints we have to impose on type class resolution.
Generally, if there is an instance declaration for $\Psi(\tau)$ and if $\tau \sub \sigma$ or $\sigma \sub \tau$ there cannot be another instance declaration for $\Psi(\sigma)$.
Otherwise, we could have the following two derivations with different witnesses if there is a subtyping derivation $\mathscr{D}$ for a witness $m : \tau \sub \sigma$:

\begin{prooftree}
  \AxiomC{$w : \instance{\Phi^<}{\tau}$}
  \RightLabel{\textsc{Decl}}
  \UnaryInfC{$i_\tau(w) : \Phi^<(\tau)$}
\end{prooftree}

\begin{prooftree}
  \AxiomC{$w' : \instance{\Phi^<}{\sigma}$}
  \RightLabel{\textsc{Decl}}
  \UnaryInfC{$i(w') : \Phi^<(\sigma)$}
  \AxiomC{$\mathscr{D}$}
  \noLine
  \UnaryInfC{$m : \tau \sub \sigma$}
  \RightLabel{\textsc{CoV}}
  \BinaryInfC{$\cov(i(w'),m) : \Phi^<(\tau)$}
\end{prooftree}

For contravariant predicates and given a derivation $\mathscr{D}'$ for a witness $m : \sigma \sub \tau$ we could have:

\begin{prooftree}
  \AxiomC{$w : \instance{\Phi^>}{\tau}$}
  \RightLabel{\textsc{Decl}}
  \UnaryInfC{$i_\tau(w) : \Phi^>(\tau)$}
\end{prooftree}

\begin{prooftree}
  \AxiomC{$w' : \instance{\Phi^>}{\sigma}$}
  \RightLabel{\textsc{Decl}}
  \UnaryInfC{$i(w') : \Phi^>(\sigma)$}
  \AxiomC{$\mathscr{D}'$}
  \noLine
  \UnaryInfC{$m : \sigma \sub \tau$}
  \RightLabel{\textsc{ContraV}}
  \BinaryInfC{$\contrav(i(w'),m) : \Phi^>(\tau)$}
\end{prooftree}

Note that in order for coherence to be satisfied, derivations do not have to be unique.
Instead of the first derivation we may also use the following additionally using the \textsc{Refl} rule:

\begin{prooftree}
  \AxiomC{$w : \instance{\Phi^<}{\tau}$}
  \RightLabel{\textsc{Decl}}
  \UnaryInfC{$i_\tau(w) : \Phi^<(\tau)$}
  \AxiomC{}
  \RightLabel{\textsc{Refl}}
  \UnaryInfC{$\refl_\tau : \tau \sub \tau$}
  \RightLabel{\textsc{CoV}}
  \BinaryInfC{$\cov(i_\tau(w),\refl_\tau) : \Phi^<(\tau)$}
\end{prooftree}

Since the derivation rules for qualified types do not branch, there will always be exactly one leaf in the derivation rule given as an instance declaration.
Thus it suffices to show that for every judgement there  may be multiple witnesses but the corresponding instance declaration is unique.
To achieve this, instances must be non-overlapping.

Additionally, in some systems with subtyping there may be overlapping types which are not themselves in a subtyping relation with each other.
In our type system, an example for this would be the types $\Singleton{0} \join \Singleton{1}$ and $\Singleton{1} \join \Singleton{2}$.
Given $\Phi(\Singleton{0} \join \Singleton{1})$ and $\Phi(\Singleton{1} \join \Singleton{2})$, the judgement $\Phi(\Singleton{1})$ could be derived in two ways.
In order to maintain coherence in this case, we have to ensure that witnesses for these judgements are derived from a common instance.
We can not, however, declare an instance for both types individually.

To ensure type class coherence, we have to take conclusions of the subtyping relation into account.
That means we have to consider the derivable instances of type class based of sub- or supertypes.
This imposes some restrictions not present in type systems based on type equalities.

  For Hindley-Milner type systems we want to avoid \emph{overlapping instances}, i.e. instances for types that can be unified. \cite{peytonjones1997type}
  Syntactically, types are overlapping if one can be substituted for the other or they can be unified.
  E.g. $\mathit{List} \; \alpha$ and $\mathit{List} \; \Nat$ are overlapping with the substitution $\sigma(\alpha) \to \Nat$.
  In Haskell specifically, it is very easy to check an overlap, albeit more restrictive: We only need to syntactically check whether two types share the same head constructor ($\mathit{List}$ in the example above).

In the context of subtyping, overlaps of types have to be expressed differently because we no longer use unification for type inference but biunification.
In order to check for an overlap of types, we have to create the intersection of those types and check whether it is empty.

E.g. given $\instance \Phi a$ and $sub \sub a$ and $a \sub sup$, we can neither have $\instance \Phi{sub}$ , nor $\instance \Phi{sup}$,
as $\Phi(sup)$ would imply $\Phi(a)$ and $\Phi(a)$ would imply $\Phi(sup)$.

In the following, I want to give a motivation on why this would be problematic.
Consider we have $\Nat \sub \Int$.
We can implement Monoid instances for both types. For natural numbers we choose multiplication as operator and accordingly 1 as neutral element.
For integers on the other hand, we might prefer to choose addition as operator and 0 as neutral element, so we can expand to monoid to a group.

Building programs on top of these instances is going to get tedious as it will often occur that the more specific \texttt{Nat} type will be inferred,
even if only want to deal with integers.
Using the append operator exposed by the Monoid typeclass, therefore may lead to unexpected behavior.
% Note for a good example, we need a good notion of type inference for this case, which is currently not implemented.

In the simple arithmetic expression $(a \oplus b) \oplus c$ $\oplus$ can have two different meanings based on the inferred types of $a,b$ and $c$.
Since type inferrence with subtyping is generally not quite obvious it may seem

  It may seem that we could choose the most specific instance. This, however, might have unexpected results.
  E.g. if we have $\mathit{NonEmptyList} \; \alpha \sub \mathit{List} \; \alpha$, we may not know during compilation whether $\mathit{NonEmptyList} \; \alpha $ or $\mathit{List} \; \alpha$ is being picked.
  ~Generally to infer the most specific type seems very hard. In this example filtering a $\mathit{NonEmptyList}$ may or may not return an empty list and we may just have to assume that it is possibly empty.
  This may lead to hard to track behaviour when using overlapping instances.

A simple example for undecidable most specific instances can be better given with record types.
If we need to resolve an instance for $C \{x : Int\}$ and we already have instances for $C \{x : Int, y : Int\}$ and $C \{x : Int, z : Int\}$ neither instance is more specific than the other.


A means to ensure type class coherence could be the following:
Since for any pair of types $\tau$ and $\sigma$ that have a common subtype $\rho$ with $\rho \sub \tau$ and $\rho \sub \sigma$ having different witnesses for $\Phi(\tau)$ and $\Phi(\sigma)$ would lead to different witnesses for $\Phi(\rho)$,
we have to ensure that $\tau \meet \sigma$ is equivalent to the empty type $\bot$.
We can do so by constructing the intersection of the type automata for $\tau$ and $\sigma$ and checking whether this automaton accepts the empty language.

% NEED TO REWORK THE PREVIOUS PARAGRAPHES

\section{Criteria for Type Class Coherence}

Resolution already guarantees that there is at most one instance resolved for any predicate application.
This instance has to be uniquely determined in order for type class coherence to be satisfied.
Thus, we need to enforce that there is at no point more than one instance in scope to witness a predicate application.

\subsection{Covariant Predicates}

Covariant predicates allow resolution 'downwards', i.e. instances for subtypes can be resolved from instances for supertypes.
We can say that any covariant instance for type $\tau$ allows resolution for a sublattice which is a \emph{downward closed set} of $\tau$. 

We therefore need to ensure globally that there is no common non-trivial subtype for which an instance can be resolved from two distinct instances.
That means the types that we declare instances for have to be \emph{non-overlapping downwards} or the two lower sets of the types declared do not contain any non-trivial types.

\begin{definition}
  Two types $\tau$ and $\sigma$ are \emph{non-overlapping downwards} iff. $\forall\rho. \rho \sub \tau$ and $\rho \sub \sigma \Rightarrow \rho$ is uninhabited.
\end{definition}

This condition is equivalent to the intersection type $\tau \meet \sigma$ being empty, which means that it has to be \emph{uninhabited}, i.e. there exist no values of that type.
In our type universe, there is no expression $e : \bot$, meaning $\bot$ is uninhabited.
Hence, deriving $\Phi(\bot)$ does not lead to any ambiguity and preserves type class coherence.
Algorithmically, this can be enforced using type automata which will be presented in the chapter \ref{ch:type-system}.

Given any covariant type class $\Phi^<$ and instances for $\Phi^<(\Singleton{0} \join \Singleton{1})$ and $\Phi^<(\Singleton{1} \join \Singleton{2})$,
the intersection of both types is equivalent to $(\Singleton{0} \join \Singleton{1}) \meet \Singleton{1} \join \Singleton{2} = \Singleton{1} \neq \bot$, so type class coherence would be violated because the instance for $\Singleton{1}$ is ambiguous.

\subsection{Contravariant Predicates}

Contravariant predicates, on the other hand, allow resolution 'upwards', i.e. instances for supertypes can be resolved from instances for subtypes.
Here, we have to enforce globally that there is no common non-trivial supertype for which an instance can be resolved from two distinct instances.
We can say that any contravariant instance for type $\tau$ allows resolution for a sublattice which is a \emph{upward closed set} of $\tau$. 
The lattice type $\top$ can be often be seen as trivial in this context because typically there are no covalues $c : \top$, i.e. we cannot define a destructor or continuation that 'destroys' a value of type $\top$.

That means the types that we declare instances for have to be \emph{non-overlapping upwards} or the two upper sets of the types declared do not contain any non-trivial types.

\begin{definition}
  Two types $\tau$ and $\sigma$ are \emph{non-overlapping upwards} iff. $\forall\rho. \tau \sub \rho$ and $\sigma \sub \rho \Rightarrow \rho$ is couninhabited.
\end{definition}

If $\top$ is couninhabited, this condition is equivalent to $\tau \join \sigma = \top$.
Enforcing this in our type system would be very restrictive.
In more sophisticated type systems, it may not even be possible to find a non-trivial type $\tau^{-1} \neq \top$, such that $\tau \join \tau^{-1} = \top$.


\subsection{Invariant Predicates}

Invariant predicates allow resolution neither 'upwards' nor 'downwards', thus no such restriction is required for them.
In other words, the set of types for which an invariant instance can be resolved for a type $\tau$ is just a singleton-set.
We only need to enforce that there is no instance for an equivalent type declared.



\section{Simplified type universes}

Consider the type universe $\mathcal{U} := \{ \Bool, \mathbb{T}, \mathbb{F} \}$.
We define the subtyping relation by $\mathbb{T} \sub \Bool$ and $\mathbb{F} \sub \Bool$.
In $\mathcal{U}$, we can conclude that $\Bool = \top$ and still have the empty type $\bot$ as a subtype of all other types.
This can be seen as a simplified version of boolean refinement types. \cite{springer}

Conversely, looking at extensible records, if our universe only consists of records and the record labels $fst : \Singleton{0}$ and $snd : \Singleton{1}$, we can set $\{\} = \top$ and $\{ fst : \Singleton{0}, snd : \Singleton{1} \} = \bot$, as there are no further extensions possible.
Formally, this would represent the type universe $\mathcal{U}' := \{ \{ fst : \Singleton{0}, snd : \Singleton{1} \}, \{ fst : \Singleton{0} \}, \{ snd : \Singleton{1} \}, \{\} \}$ with the subtyping relation defined for extensible records. %TODO fix, extensible records are not defined yet

Using $\Showable$ as example for covariant and $\Defaultable$ as example for contravariant predicates, fig. \ref{fig:satisfiability-example} shows how refinements and extensions behave when resolving co- and contravariant predicates and exhibits an interesting symmetry:
Refinements for covariant predicates behave like extensions for contravariant predicates, whereas refinements for contravariant predicates behave like extensions for covariant predicates.

\begin{landscape}

\begin{figure}[ht]
\begin{center}
  \begin{tabular}{| c | c c |}
  \hline
        & Covariant instance & Satisfies (Lower set) \\
  \hline
   Data & $\Showable(\Bool)$      & $\Showable(\mathbb{T}), \Showable(\mathbb{F}), \Showable(\bot)$ \\
        & $\Showable(\mathbb{T})$ & $\Showable(\bot)$ \\
        & $\Showable(\mathbb{F})$ & $\Showable(\bot)$ \\
        & $\Showable(\bot)$       &                   \\
   \hline
   Compatible & \multicolumn{2}{c|}{$\{\Showable(\Bool)\}, \{ \Showable(\mathbb{T}) \}, \{ \Showable(\mathbb{F}) \}, \{ \Showable(\mathbb{T}), \Showable(\mathbb{F}) \}, \{\Showable(\bot)\} $}  \\
   \hline 
   Codata & $\Showable(\{ fst : \Singleton{0}, snd : \Singleton{1} \})$ & \\
          & $\Showable(\{ fst : \Singleton{0} \})$                      & $\Showable(\{ fst : \Singleton{0}, snd : \Singleton{1} \}$ \\
          & $\Showable(\{ snd : \Singleton{1} \})$                      & $\Showable(\{ fst : \Singleton{0}, snd : \Singleton{1} \})$ \\
          & $\Showable(\{ \})$                                          & $\Showable(\{ fst : \Singleton{0}, snd : \Singleton{1} \}), \Showable(\{ fst : \Singleton{0} \}), \Showable(\{ snd : \Singleton{1} \})$ \\
   \hline
   Compatible & \multicolumn{2}{c|}{$\{\Showable(\{ fst : \Singleton{0}, snd : \Singleton{1} \})\}, \{\Showable(\{ fst : \Singleton{0} \})\}, \{\Showable(\{ snd : \Singleton{1} \})\}, \{\Showable(\{ \})\}$}  \\
   \hline 
  \end{tabular}
\end{center}

\begin{center}
  \begin{tabular}{| c | c c |}
  \hline
        & Contravariant instance & Satisfies (Upper set) \\
  \hline
   Data & $\Defaultable(\Bool)$ & \\
        & $\Defaultable(\mathbb{T})$ & $\Defaultable(\Bool)$ \\
        & $\Defaultable(\mathbb{F})$ & $\Defaultable(\Bool)$ \\
        & $\Defaultable(\bot)$ & $\Defaultable(\Bool), \Defaultable(\mathbb{T}), \Defaultable(\mathbb{F})$ \\
   \hline
   Compatible & \multicolumn{2}{c|}{$\{\Defaultable(\Bool)\}, \{ \Defaultable(\mathbb{T}) \}, \{ \Defaultable(\mathbb{F}) \}, \{\Defaultable(\bot)\} $}  \\
   \hline 
   Codata & $\Defaultable(\{ fst : \Singleton{0}, snd : \Singleton{1} \})$ & $\Defaultable(\{ fst : \Singleton{0} \}), \Defaultable(\{ snd : \Singleton{1} \}), \Defaultable(\{ \})$ \\
          & $\Defaultable(\{ fst : \Singleton{0} \})$ & $\Defaultable(\{ \})$ \\
          & $\Defaultable(\{ snd : \Singleton{1} \})$ & $\Defaultable(\{ \})$ \\
          & $\Defaultable(\{ \})$ & \\
   \hline
   Compatible & \multicolumn{2}{c|}{$\{\Defaultable(\{ fst : \Singleton{0}, snd : \Singleton{1} \})\}, \{\Defaultable(\{ fst : \Singleton{0} \})\}, \{\Defaultable(\{ snd : \Singleton{1} \})\}$} \\
              & \multicolumn{2}{c|}{$\{\Defaultable(\{ fst : \Singleton{0} \}), \Defaultable(\{ snd : \Singleton{1} \})\}, \{\Defaultable(\{ \})\}$}  \\
   \hline 
  \end{tabular}
\end{center}
\caption{Satsifiability for co- and contravariant predicates for data and codata. \\ Reflexivity of satisfiability is implicit. }
\label{fig:satisfiability-example}

\end{figure}

\end{landscape}

Observing these simple closed type universes, we can conclude that instances for types can only be defined if they are unrelated or non-overlapping.
That means for each type class we can either define an instance for the whole universe (corresponding to $\top$ or $\bot$) or one or more for more specific types (i.e. record types with exactly one label or refined nominal types like $\mathbb{T}$ and $\mathbb{F}$).

When looking at more sophisticated type universes however, there are some types that are seemingly unrelated but still complicate things when we want to ensure coherence.
E.g. if we have instances $\Showable(\Bool)$ and $\Showable(\Nat)$, it is obvious that their intersection is empty  because $\Bool \meet \Nat = \bot$.
Thus declaring these instances in the same scope cannot cause the resoltution of incoherent instances.

The story is different, however, when we look at contravariant predicates.
Given $\Defaultable(\Bool)$ and $\Defaultable(\Nat)$, the union $\Bool \join \Nat$ generally is not equivalent to $\top$. % Even if it were there may be problems...
Thus, if we have to resolve an instance for $\Defaultable(\Bool \join \Nat)$ the resolution is going to be ambiguous.
In practise, we would have two default values to choose from whereas it is unclear which one should be taken.

For practical programming languages, we can check the coherence of instances without doing any resolution.
For contravariant type classes, this approach seems too restrictive because we would only ever be able to declare one instance.
In this case, it is more useful to fail during resolution when ambiguous instances can be resolved.

There are two cases where we can have two distinct instances declared at the same time without violating type class coherence:
On the one hand we can declare non-overlapping instances for covariant type classes of data, i.e. $\Showable(\mathbb{T})$ and $\Showable(\mathbb{F})$ which both let us only resolve the constraint $\Showable(\bot)$, which is unproblematic as $\bot$ uninhabited.
On the other hand we can declare non-overlapping instances for contravariant type classes of codata, i.e. $\Defaultable(\{fst : \Singleton{0}\})$ and $\Defaultable(\{snd : \Singleton{1}\})$ which both let us only resolve the constraint $\Defaultable(\top)$ (which is equivalent to $\Defaultable(\{\})$), which is unproblematic as $\top$ couninhabited.
The record containing all fields in this universe is couninhabited because there are no field selectors which play the role of destructors for records.


\section{Coherence for Sublattices}

The lattice of types can be subdivided into sublattices.
Thus, we may define coherence for sublattices in isolation.
Using the notion of upward and downward closed sets in relation to sublattices, we have to identify uninhabited and couninhabited types and check for overlaps of downward or upward closed sets.
If we have, for example, a contravariant instance for a record type $\{ \ell_2 : \tau_2, \ell_3 : \tau_3 \}$, looking at the lattice, we can identify its upset as shown in fig. \ref{fig:upsets}.
Knowing the upset and the couninhabited $\{\}$, we can identify compatible and incompatible instances in the sublattice.
In this case, any instance for records that do not contain $\ell_2$ or $\ell_3$ are compatible with this instance.

\begin{figure}
  \centering
  \begin{tikzcd}
    & {\color{blue}\{ \}} \arrow[ld] \arrow[rd]                     &                                      \\
    {\color{blue}\{ fst : \Singleton{0} \}} \arrow[rd] &                                                 & \{ snd : \Singleton{1} \} \arrow[ld] \\
    & \{ fst : \Singleton{0}, snd : \Singleton{1} \}  &
  \end{tikzcd}
  \caption{Sublattice of record types. In {\color{blue}blue}: Upward closed set of $\{ fst : \Singleton{1} \}$}
  \label{fig:upset}
\end{figure}

\begin{figure}
  \centering
  \begin{tikzcd}
    &                                                      & {\color{blue}\{\}} \arrow[lld] \arrow[ld] \arrow[d] \arrow[rrd]   &       &                                     \\
    \{ \ell_1 : \tau_1 \} \arrow[d] \arrow[rrd]           & {\color{blue}\{ \ell_2 : \tau_2 \}} \arrow[d] \arrow[ld]           & {\color{blue}\{ \ell_3 : \tau_3 \}} \arrow[d] \arrow[ld]          & \dots & \{ \ell_n : \tau_n \} \arrow[llddd] \\
    {\{ \ell_1 : \tau_1, \ell_2 : \tau_2 \}} \arrow[rrdd] & {\color{blue}\{ \ell_2 : \tau_2, \ell_3 : \tau_3 \}} \arrow[rdd] & {\{ \ell_1 : \tau_1, \ell_3 : \tau_3 \}} \arrow[dd] & \dots &                                     \\
    &                                                      & \dots                                               &       &                                     \\
    &                                                      & {\{ \ell_1 : \tau_1, \dots \ell_n : \tau_n \}}      &       &
  \end{tikzcd}
  \caption{Sublattice of record types. In {\color{blue}blue}: Upward closed set of $\{ \ell_2 : \tau_2, \ell_3 : \tau_3 \}$}
  \label{fig:upsets}
\end{figure}

Generally, the compability of instances in a sublattice is more critical than in the whole lattice.
We may want to be restrictive in an open lattice like that of extensible records but more liberal in a lattice of nominal types.

E.g. if we have contravariant instances for $\Defaultable(\Bool)$ and $\Defaultable(\Nat)$, they would be incompatible because $\Defaultable(\Bool \join \Nat)$ can be resolved from both instances.
In such cases we could restrict type class constraints to only be applied to nominal types without explicit unions and intersections.