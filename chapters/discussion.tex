%%%%%%%%%%%%%%%%%%%%%%%%%%%%%%%%%%%%%%%%%%%%%%%%%%%%%%%%%%%%%%%%%%%%
% Diskussion und Ausblick
%%%%%%%%%%%%%%%%%%%%%%%%%%%%%%%%%%%%%%%%%%%%%%%%%%%%%%%%%%%%%%%%%%%%

\chapter{Discussion and Outlook}
\label{ch:discussion}

\section{Future Work}

\subsection{Multi Parameter Type Classes}

The concrete implementation may impose further restrictions on instance resolution.

Our language differentiattes covariant and contravariant type classes.
This distinction is made for the type variable declared in the class declaration.

In a covariant type class, type variables may only occur on covariant positions in the class methods signatures.
A simple example for a covariant type class is the \mintinline{Haskell}{Show} class:

\begin{minted}{Haskell}
  class Show +a where
    show :: a -> String
\end{minted}

Dually in a contravariant type class, type variables may only occur on contravariant positions in the class methods signatures.
A simple example for a contravariant type class is the \mintinline{Haskell}{Read} class:

\begin{minted}{Haskell}
  class Read -a where
    read :: Read -> a
\end{minted}

This distinction imposes a problem when we want to implement type classes in which the type variable may occur both in a covariant and contravariant position.

\begin{minted}{Haskell}
  class Semigroup a where
    mappend :: a -> a -> a
\end{minted}

One way to solve this problem is by distinguishing covariant and contravariant type variables.
We can do so by introducing multi parameter type classes:

\begin{minted}{Haskell}
  class Semigroup +a -b where
    mappend :: a -> a -> b
\end{minted}

Instead of defining an instance for \mintinline{Haskell}{Semigroup Nat}, we would then have to define an instance for \mintinline{Haskell}{Semigroup Nat Nat}.
The latter seems to be less intuitive because semigroups are not a relation between types but a property for just one type (the operator has to map two elements from one set into the same set).
Since such cases may occur in many other classes (e.g. Num, Monad) it may be helpful to define syntactic sugar for type classes with mixed variance.
Then, the less intuitive implementation as multi parameter type classes could be hidden on the surface syntax.

Discuss instance chains:
We can relax this constraint by defining an explicit order in which instances should be picked/resolved.
\cite{morris2010instance}
