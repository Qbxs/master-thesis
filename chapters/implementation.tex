\chapter{Implementation}
\label{ch:implementation}

\section{Example}

We introduce the type class $\mathit{Summable}$ of types which can be summed up or folded into a natural number.

\begin{flalign*}
& \mathbf{class} \; \mathit{Summable}(+a)\{ \\
& \;\;\; \mathbf{sum} : a  \to \Nat \\
& \}
\end{flalign*}

We can implement this type class for pairs of natural  numbers by simply adding up the components.

\begin{flalign*}
& \mathbf{instance} \; \mathit{Summable} \; \mathit{Pair}(\Nat,\Nat) \{ \\
& \;\;\; \mathbf{sum} := \lambda p. \mathbf{case} \; p \; of \\
& \;\;\;\;\; (n,m) \Rightarrow n+m \\
& \}
\end{flalign*}

Given a definition for non-empty lists, we can also use an instance recursively to sum up its elements:


\begin{flalign*}
& \mathbf{data} \; \mathit{NonEmpty} \{ \\
& \;\;\; \mathit{Singleton}(\Nat) \\
& \;\;\; \mathit{Cons}(\Nat, \mathit{NonEmpty})\\
& \}
\end{flalign*}

\begin{flalign*}
& \mathbf{instance} \; \mathit{Summable} \; \mathit{NonEmpty} \{ \\
& \;\;\; \mathbf{sum} := \lambda l. \mathbf{case} \; l \; of \\
& \;\;\;\;\; \mathit{Singleton}(n) \Rightarrow n \\
& \;\;\;\;\; \mathit{Cons}(n, ns) \Rightarrow n + (\mathbf{sum}[\mathit{NonEmpty}] \; ns) \\
& \}
\end{flalign*}

Similarily, we can define a \mathit{Summable} instance for leaf-oriented trees.

\begin{flalign*}
& \mathbf{data} \; \mathit{NatTree} \{ \\
& \;\;\; \mathit{Leaf}(\Nat) \\
& \;\;\; \mathit{Branch}(\mathit{NatTree}, \mathit{NatTree})\\
& \}
\end{flalign*}

\begin{flalign*}
& \mathbf{instance} \; \mathit{Summable} \; \mathit{NatTree} \{ \\
& \;\;\; \mathbf{sum} := \lambda t. \mathbf{case} \; t \; of \\
& \;\;\;\;\; \mathit{Leaf}(n) \Rightarrow n \\
& \;\;\;\;\; \mathit{Branch}(l, r) \Rightarrow (\mathbf{sum}[\mathit{NatTree}] \; l) + (\mathbf{sum}[\mathit{NatTree}] \; r) \\
& \}
\end{flalign*}

\section{Duo}

\begin{lstlisting}[style=duostyle, label=code:duo-example, caption={Example duo code}]


    module Class.NatFoldable;

    import Data.Peano;
    import Data.Tensor;

    -- | The class of nonempty types that can be folded into a natural number.
    class NatFoldable(+a : CBV){
        Fold(a, return Nat)
    };

    instance natTensorFoldable : NatFoldable Tensor(Nat, Nat) {
        Fold(p, k) => case p of {
            MkTensor(n, m) => add n m >> k
        }
    };

    -- | A non-empty list of natural numbers.
    data NatList {
        Singleton(Nat),
        Cons(Nat, NatList)
    };

    instance natListFoldable : NatFoldable NatList {
        Fold(l, k) => case l of {
            Singleton(n) => n >> k,
            Cons(n, ns) => Fold(ns, mu x. add x n >> k)
        }
    };

    -- | A leaf-oriented tree of natural numbers.
    data NatTree {
        Leaf(Nat),
        Branch(NatTree, NatTree)
    };

    instance natTreeFoldable : NatFoldable NatTree {
        Fold(t, k) => case t of {
            Leaf(n) => n >> k,
            Branch(l, r) => add (mu k1. Fold(l, k1)) (mu k2. Fold(r, k2)) >> k
        }
    };

\end{lstlisting}


% Outline the duo language


\section{Pipeline}

% explain, how to get from the surface syntax to a typechecked runnable program
% parsing, resolution, desugaring, constraint generation, solving, instance resolution

\section{Remarks}

% Outline the interaction between type classes and miscelleanous language features
% e.g. refinement types
