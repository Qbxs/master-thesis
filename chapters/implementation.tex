\chapter{Implementation}
\label{ch:implementation}

\section{Representations of Type Class Witnesses} % not quite correct
\label{sec:representations}

For an implementation, we need some representation of witnesses which is going to be used for the execution of programs using type class methods.
Since type classes hide the concrete implementation of their methods, we need a means to extract this concrete implementation when needed.
There are essentialy two ways to implement type class overloading in practise: Intensional type analysis which dispatches a type representation at runtime and dictionary passing which treats witnesses as records or dictionaries which can be resolved at compile time.

\subsection{Intensional type analysis}
\label{sec:intensional-analysis}

Instead of passing around dictionaries, we can also resolve the relevant instance at runtime by dispatching on the term's type.
This requires our language to tag terms with some representation of their type at runtime.
For the \mintinline{Haskell}{Show} type class the definition of \mintinline{Haskell}{show} could look like this:

\begin{minted}{Haskell}
  show t x = case t of
    Bool -> boolShow x
    Int -> intShow x
\end{minted}

In the context of subtyping we would not require type equality but for \mintinline{Haskell}{t} to be a subtype of some type for which an instance is defined.
This also renders intensional type analysis more difficult:
We can not simply match on a type (or its encoded representation \cite{weirich2000}) because we have to essentially solve the problem of subtyping between the type at hand and (at worst case) for each type for which an instance declaration is in scope.
So, intensional type analysis may introduce a dramatic overhead in the presence of subtyping.

This overhead seems natural when we want to implement a dynamically type checked language.
However, type class resolution should already guarantee us static type safety, so additionally using type tags for dynamic type checking or type analysis seems superfluous.
The only case for which intensional type analysis appears superior is for the case of union and intersection types.


\subsection{Dictionary Passing}
\label{sec:dictionaryPassing}

One typical way of implementing type classes is using \emph{dicitonary passing style}. \cite{kiselyov}
To examine this, we look at type classes in Haskell.
We use an isomorphism between type classes and records:
E.g. the type class \mintinline{Haskell}{Eq} defined as

\begin{minted}{Haskell}
  class Eq a where
    eq :: a -> a -> Bool
    neq :: a -> a -> Bool
\end{minted}

can be translated to a record type that preserves the structure of the class:

\begin{minted}{Haskell}
  data DictEq a =
    DictEq { eq :: a -> a -> Bool,
             neq :: a -> a -> Bool }
\end{minted}

Instances as witnesses of type classes can be translated accordingly to values of dictionaries.
E.g.

\begin{minted}{Haskell}
  instance Eq Int where
    eq = intEq
    neq = not . intEq  
\end{minted}

can be translated to a value of type \mintinline{Haskell}{DictEq Int}:

\begin{minted}{Haskell}
  intEqDict = DictEq { eq = intEq,
                       neq = not . intEq }
\end{minted}

Here, the instances of a type class are carried in a dictionary that has to be passed as an additional argument to method calls.
For example the term

\begin{minted}{Haskell}
  show 5
\end{minted}

would be compiled to

\begin{minted}{Haskell}
  show showIntDict 5
\end{minted}

where \mintinline{Haskell}{showIntDict} is a dictionary that provides the relevant defintion of \mintinline{Haskell}{show :: Int -> String} for the instance of \mintinline{Haskell}{Show Int}.
This allows us to "compile away" the overhead that type classes introduce to the surface language because type inference can fill in the correct dictionary that is ought to be used.

Generic constrained functions like

\begin{minted}{Haskell}
  emphasize :: (Show a) => a -> String
  emphasize x = show x ++ "!"
\end{minted}

would simply pass around the dictionary and be compiled to something like:

\begin{minted}{Haskell}
  emphasize :: ShowDict a -> a -> String
  emphasize dict x = show dict x ++ "!"
\end{minted}

If we were to consider composed instances for e.g. union types, as in the following term:

\begin{minted}{text}
  show (if b then 42 :: Int else "Hello" :: String)
\end{minted}

The inferred type of this expression should be \mintinline{text}{Int \/ String}.
To represent this, we would need two dictionaries: One for \mintinline{text}{Int} and one for \mintinline{text}{String} because it is undecidable at compile time which dictionary is going to be used at runtime.
However, since our resolution rules do not permit this, passing a single dictionary will suffice.

With subtyping, these dictionaries have to preserve subtyping properties.
That means, if $\tau \sub \sigma$ and we have a dictionary $\mathit{Dict}(\sigma)$, then $\mathit{Dict}(\tau) \sub \mathit{Dict}(\sigma)$.
\todo{how does it work with subtyping?}


\section{Example}

In the following, we want to show how the presented type classes can be used in practise.
We introduce the type class $\mathit{Summable}$ of types which can be summed up or folded into a natural number.

\begin{flalign*}
     & \mathbf{class} \; \mathit{Summable}(+a)\{ \\
     & \;\;\; \mathbf{sum} : a  \to \Nat         \\
     & \}
\end{flalign*}

We can implement this type class for the nominal type of integers $\Int$ by just returning the argument.

\begin{flalign*}
     & \mathbf{instance} \; \mathit{Summable} \; \Int \{ \\
     & \;\;\; \mathbf{sum} := \mathbf{id}               \\
     & \}
\end{flalign*}

Since the class is covariant, we can also use it for more specific types, i.e. types that are subtypes of $\Int$.
For example, the type class method invocation in $\mathbf{sum} \; (1 : \Nat)$ can be done via the instance declared for $\Int$.

Given a definition for non-empty lists, we can also use an instance recursively to sum up its elements:


\begin{flalign*}
     & \mathbf{data} \; \mathit{NonEmpty} \{         \\
     & \;\;\; \mathit{Singleton}(\Nat)               \\
     & \;\;\; \mathit{Cons}(\Nat, \mathit{NonEmpty}) \\
     & \}
\end{flalign*}

\begin{flalign*}
     & \mathbf{instance} \; \mathit{Summable} \; \mathit{NonEmpty} \{                          \\
     & \;\;\; \mathbf{sum} := \lambda l. \mathbf{case} \; l \; of                              \\
     & \;\;\;\;\; \mathit{Singleton}(n) \Rightarrow n                                          \\
     & \;\;\;\;\; \mathit{Cons}(n, ns) \Rightarrow n + (\mathbf{sum}[\mathit{NonEmpty}] \; ns) \\
     & \}
\end{flalign*}

Similarily, we can define a $\mathit{Summable}$ instance for leaf-oriented trees.

\begin{flalign*}
     & \mathbf{data} \; \mathit{NatTree} \{                       \\
     & \;\;\; \mathit{Leaf}(\Nat)                                 \\
     & \;\;\; \mathit{Branch}(\mathit{NatTree}, \mathit{NatTree}) \\
     & \}
\end{flalign*}

\begin{flalign*}
     & \mathbf{instance} \; \mathit{Summable} \; \mathit{NatTree} \{                                                              \\
     & \;\;\; \mathbf{sum} := \lambda t. \mathbf{case} \; t \; of                                                                 \\
     & \;\;\;\;\; \mathit{Leaf}(n) \Rightarrow n                                                                                  \\
     & \;\;\;\;\; \mathit{Branch}(l, r) \Rightarrow (\mathbf{sum}[\mathit{NatTree}] \; l) + (\mathbf{sum}[\mathit{NatTree}] \; r) \\
     & \}
\end{flalign*}

\todo{add examples for usage of type class methods}

\section{Duo}

\emph{Duo} is a research language developed at the University of T\"ubingen. \cite{duo}
It is a statically typed functional language with an emphasis on duality.
Unlike most known functional languages its type system is based on sequent calculus style deduction rather than natural deduction.
Most notably, computation happens as a "cut" between producer and consumer terms using the \lstinline{>>} operator.

E.g. in the program in listing \ref{code:duo-example} the producer \lstinline{True} is matched against a consumer, i.e. a pattern-match on booleans, which simply exits the program.

\begin{lstlisting}[style=duostyle, label=code:duo-example, captionpos=b, caption={Example duo code}]
def cmd foo := True >> case {
    True => #ExitSuccess;
    False => #ExitFailure
};
\end{lstlisting}

The nature of \emph{duo} has some special implications for the implementation of type classes as well:
In class declarations, we have a kind signature for the type variable bound by the declaration.
Here, we also define the variance of the variable.
We implement methods as commands.

\begin{lstlisting}[style=duostyle, label=code:duo-class-decl, captionpos=b, caption={Type class declaration  in \emph{duo}}]

class Summable(+a : CBV){
    Sum(a, return Nat)
};

\end{lstlisting}

An instance is named, which makes it easier to reference internally.
In the instance we have to implement each method declared in the class declaration as a command.

\begin{lstlisting}[style=duostyle, label=code:duo-nat-list, captionpos=b, caption={Instance declaration for natural numbers in \emph{duo}}]
instance natSummable : Summable Nat {
    Sum(n, k) => n >> k
};
\end{lstlisting}

\emph{Duo} supports the usual algebraic data types. Non-empty lists of natural numbers can be defined like this:

\begin{lstlisting}[style=duostyle, label=code:duo-nat-list-two, captionpos=b, caption={Data and instance declaration for non-empty lists of in \emph{duo}}]

data NatList {
    Singleton(Nat),
    Cons(Nat, NatList)
};

instance natListSummable : Summable NatList {
    Sum(l, k) => case l of {
        Singleton(n) => n >> k,
        Cons(n, ns) => Sum(ns, mu x. add x n >> k)
    }
};

\end{lstlisting}

We can also add more instances. For example, for non-empty leaf-oriented binary trees.

\begin{lstlisting}[style=duostyle, label=code:duo-tree-instance, captionpos=b, caption={Type class example  in \emph{duo}}]

data NatTree {
    Leaf(Nat),
    Branch(NatTree, NatTree)
};

instance natTreeSummable : Summable NatTree {
    Sum(t, k) => case t of {
        Leaf(n) => n >> k,
        Branch(l, r) => add (mu k1. Sum(l, k1))
                            (mu k2. Sum(r, k2)) >> k
    }
};

\end{lstlisting}

% Outline the duo language
Several checks have been implemented to ensure type class coherence in \emph{duo}:
For modularity, we use the common approach to disallow orphan instances. \cite{Kilpatrick2019-cy}
That means, an instance declaration is required to be placed in the same module as the data or class declaration.
When declaring an instance, we check for all instances of the same type class in scope whether the overlap with the type the instance is being declared for is empty.


\section{Refinement Types}
\label{sec:refinement-types}
\todo{unfinished section, maybe remove}
Refinement types can be viewed as a specialisation of nominal types.
Using them, we can impose additional constraints on types and therefore obtain a more precise typing judgement. \cite{springer} %FIXME wrong citation
In \emph{duo} refinement types are declared very similarily to nominal types:

\begin{lstlisting}[style=duostyle, label=code:duo-refinement-declaration, captionpos=b, caption={Refinement type of peano numbers in \emph{duo}}]

refinement data Nat {
    Z,
    S(Nat)
};

\end{lstlisting}

The type inference algorithm in duo is expanded to refinement types and therefore able to infer very precise types for terms. \cite{binder22refinement}
As we can see in listing \ref{code:duo-refinement-example} the type for the simple term \lstinline{Z} reflects that it can only be constructed by \lstinline{Z}, therefore we know from looking at the type alone, that the only value of this term can be zero.
We also need recursive types to allow recursive refinement types as seen in \lstinline{double}.
Here we can see that any natural number is permitted as input but only numbers constructed by an even number of \lstinline{S} constructors will be in the output.
In this case refinement types can assure us that this function will only produce even numbers.

\begin{lstlisting}[style=duostyle, label=code:duo-refinement-example, captionpos=b, caption={Type inference for refinement types in \emph{duo}}]

def rec prd zero : < Nat | Z > := Z;

def rec prd double : (rec a.(<Nat | Z, S(a)>))
                  -> (rec a.(<Nat | Z, S(< Nat | S(a)>)>))
    := \n => case n of {
      Z => Z,
      S(n) => S(S(double n))
    };
    
\end{lstlisting}

\todo{bring type classes and refinement types together}