\chapter{Implementation}
\label{ch:implementation}

\section{Example}

\begin{lstlisting}


    module Class.NatFoldable;

    import Data.Peano;
    import Data.Tensor;
    
    -- | The class of nonempty types that can be folded into a natural number.
    class NatFoldable(+a : CBV){
        Fold(a, return Nat)
    };
    
    instance natTensorFoldable : NatFoldable Tensor(Nat, Nat) {
        Fold(p, k) => case p of {
            MkTensor(n, m) => add n m >> k
        }
    };
    
    -- | A non-empty list of natural numbers.
    data NatList {
        Singleton(Nat),
        Cons(Nat, NatList)
    };
    
    instance natListFoldable : NatFoldable NatList {
        Fold(l, k) => case l of {
            Singleton(n) => n >> k,
            Cons(n, ns) => Fold(ns, mu x. add x n >> k)
        }
    };
    
    -- | A leaf-oriented tree of natural numbers.
    data NatTree {
        Leaf(Nat),
        Branch(NatTree, NatTree)
    };
    
    instance natTreeFoldable : NatFoldable NatTree {
        Fold(t, k) => case t of {
            Leaf(n) => n >> k,
            Branch(l, r) => add (mu k1. Fold(l, k1)) (mu k2. Fold(r, k2)) >> k
        }
    };

\end{lstlisting}

\section{Duo}

% Outline the duo language


\section{Pipeline}

% explain, how to get from the surface syntax to a typechecked runnable program
% parsing, resolution, desugaring, constraint generation, solving, instance resolution

\section{Remarks}

% Outline the interaction between type classes and miscelleanous language features
% e.g. refinement types