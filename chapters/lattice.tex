\chapter{The Lattice of Types}
\label{ch:lattice}

A partially ordered set $(L , \leq)$ is called a lattice if it is both a join- and a meet-semilattice, i.e. each two-element subset $\{a,b\}\subseteq L$ has a join (i.e. least upper bound, denoted by $a \join b$) and dually a meet (i.e. greatest lower bound, denoted by $a \meet b$).
A lattice of types is a type universe $\mathcal{U}$ of types and a subtyping relation $\sub$ playing the role of a partial order.
It is, furthermore, a bounded lattice due to the types $\top$ and $\bot$ which satisfy $\bot \sub \tau$ and $\tau \sub \top$ for all types $\tau$ in $\mathcal{U}$.
In the following, we introduce a simple language with a few nominal types to describe a type system with subtyping.

\section{Types}
\label{sec:types}

We use the following syntax of types (fig. \ref{fig:type-syntax}) with the simple types of natural numbers $\Nat$, of integers $\Int$ and the type of booleans $\Bool$.
Additionally, we introduce the singleton types $\Singleton{0}, \Singleton{1}, \Singleton{2}, \dots$ which each are only inhabited by the term in their index, e.g. $1 : \Singleton{1}$.
We extend the syntax of types with extensible record types. \cite{leijen2005extensible}
Extensible records are a form of codata as they are defined via thair destructors or field labels.
Because they are extensible, they are in an interesting subtyping relation, i.e. we can add fields which make a record more specific but still a subtype of another one.
E.g. the record type $\{ x : \Singleton{1}, y : \Nat \}$ is a subtype of the type $\{ y : \Nat \}$ as it simply represents the same type with one addtional field $x$. \\
Furthermore, for subtyping we consider the lattice types constructed by $\top, \bot, \meet, \join$.
$\top$ stands for the greatest type which is a supertype of all other types.
Dually, $\bot$ is the smallest type which is a subtype of all other types.
Types constructed by the $\join$ and $\meet$ operators stand for the union and intersection types, respectively.
\todo{clarify recursive types. needed?}

\begin{figure}[ht]
  \begin{flalign*}
    ts          := & \; \forall \alpha_1 \dots \alpha_n. \tau                                 & \textit{Type schemes}             \\
    \tau,\sigma := & \; \Bool \; | \; \Nat \; | \; \Int \; | \; \String                       & \textit{Simple types}             \\
                   & \; \Singleton{0} \; | \; \Singleton{1} \; | \; \Singleton{2} \; | \dots  & \textit{Singleton types}          \\
                   & \; \top \; | \; \bot \; | \; \tau \meet \sigma \; | \; \tau \join \sigma & \textit{Lattice types}            \\
                   & \; \tau \to \sigma                                                       & \textit{Function types}           \\
                   & \; \{ \ell_i : \tau_i \}                                                 & \textit{Record types}             \\
    % remove recursive types?
                   & \; \mu\alpha.\tau                                                        & \textit{Recursive types}          \\
                   & \; \alpha                                                                & \textit{Recursive type variables}
  \end{flalign*}
  \caption{The syntax of types}
  \label{fig:type-syntax}
\end{figure}
                
\subsection{Subtyping Rules}
\label{sec:subtyping-rules}

We define primitive subtyping rules for nominal types. (Fig. \ref{fig:prim-sub-rules}).
We want to express that every natural number is also an integer and therefore introduce a corresponding rule, as well as a rule for the singleton types $\Singleton{0}, \Singleton{1}, \Singleton{2}$ to be subtypes of $\Nat$:

\begin{figure}[ht]
  \begin{center}
    \AxiomC{}
    \RightLabel{\textsc{NatPrim}}
    \UnaryInfC{$\Nat \sub \Int$}
    \DisplayProof
    {\hskip.2in}
    \AxiomC{}
    \RightLabel{\textsc{SingletonPrim}}
    \UnaryInfC{$\Singleton{0}, \Singleton{1}, \Singleton{2}, \dots \sub \Nat$}
    \DisplayProof
  \end{center}

  \caption{Primitve subtyping rules}
  \label{fig:prim-sub-rules}
\end{figure}

The subtyping lattice is defined by the following derivation rules where $\tau$, $\sigma$ and $\rho$ are meta variables for types. (fig. \ref{fig:subtyping})
These rules directly reflect certain facts about the lattice types:
$\top$ is a supertype of all other types, $\bot$ conversly is a subtype of all other types.
Union and intersection types are dual to each other.
Functions are contravariant in their input type, thus flipping the subtyping relation, and covariant in their output type.

% elim rules are derivable
\begin{figure}[ht]
  % refl
  \begin{prooftree}
    \AxiomC{}
    \RightLabel{\textsc{Refl}}
    \UnaryInfC{$\tau \sub \tau$}
  \end{prooftree}

  % trans
  \begin{prooftree}
    \AxiomC{$\tau \sub \sigma$}
    \AxiomC{$\sigma \sub \rho$}
    \RightLabel{\textsc{Trans}}
    \BinaryInfC{$\tau \sub \rho$}
  \end{prooftree}

  % top/bot
  \begin{center}
    \AxiomC{}
    \RightLabel{\textsc{Top}}
    \UnaryInfC{$\tau \sub \top$}
    \DisplayProof
    {\hskip.2in}
    \AxiomC{}
    \RightLabel{\textsc{Bot}}
    \UnaryInfC{$\bot \sub \tau$}
    \DisplayProof
  \end{center}

  % join intro
  \begin{center}
    \AxiomC{$\tau \sub \sigma$}
    \RightLabel{\textsc{JoinI}$_1$}
    \UnaryInfC{$\tau \sub \sigma \join \rho$}
    \DisplayProof
    {\hskip.2in}
    \AxiomC{$\tau \sub \sigma$}
    \RightLabel{\textsc{JoinI}$_2$}
    \UnaryInfC{$\tau \sub \rho \join \sigma$}
    \DisplayProof
  \end{center}

  \begin{prooftree}
    \AxiomC{$\tau \sub \rho$}
    \AxiomC{$\sigma \sub \rho$}
    \RightLabel{\textsc{JoinI}}
    \BinaryInfC{$\tau \join \sigma \sub \rho$}
  \end{prooftree}

  % meet intro
  \begin{center}
    \AxiomC{$\tau \sub \rho$}
    \RightLabel{\textsc{MeetI}$_1$}
    \UnaryInfC{$\tau \meet \sigma \sub \rho$}
    \DisplayProof
    {\hskip.2in}
    \AxiomC{$\sigma \sub \rho$}
    \RightLabel{\textsc{MeetI}$_2$}
    \UnaryInfC{$\tau \meet \sigma \sub \rho$}
    \DisplayProof
  \end{center}

  \begin{prooftree}
    \AxiomC{$\tau \sub \rho$}
    \AxiomC{$\tau \sub \sigma$}
    \RightLabel{\textsc{MeetI}}
    \BinaryInfC{$\tau \sub \sigma \meet \rho$}
  \end{prooftree}

  \begin{prooftree}
    \AxiomC{$\tau' \sub \tau$}
    \AxiomC{$\sigma \sub \sigma'$}
    \RightLabel{\textsc{Func}}
    \BinaryInfC{$\tau \to \sigma \sub \tau' \to \sigma'$}
  \end{prooftree}

  \begin{prooftree}
    \AxiomC{$\forall j. \tau_j \sub \sigma_j$}
    \AxiomC{$\{ \overline{\ell_j} \} \subseteq \{ \overline{\ell_i} \}$}
    \RightLabel{\textsc{Extend}}
    \BinaryInfC{$\{ \overline{\ell_i : \tau_i} \} \sub \{ \overline{\ell_j : \sigma_j} \}$}
  \end{prooftree}

  % recursive types
  \begin{center}
    \AxiomC{}
    \RightLabel{\textsc{Rec}$_1$}
    \UnaryInfC{$\mu\rho.\tau \sub \tau [\mu\rho.\tau\ / \rho]$}
    \DisplayProof
    {\hskip.2in}
    \AxiomC{}
    \RightLabel{\textsc{Rec}$_2$}
    \UnaryInfC{$\tau[\mu\rho.\tau / \rho] \sub \mu\rho.\tau$}
    \DisplayProof
  \end{center}

  \caption{Subtyping rules}
  \label{fig:subtyping}
\end{figure}

As an example, we can show that an extended record of type $\{ \mathit{x} : \Nat, \mathit{y} : \Singleton{0} \}$  is a subtype of the simpler record type $\{ \mathit{x} : \Int \}$. (fig. \ref{fig:subtyping-example})

\begin{figure}[ht]
  \begin{center}
    \AxiomC{}
    \RightLabel{\textsc{NatPrim}}
    \UnaryInfC{$\Nat \sub \Int$}
    \AxiomC{$\{ \mathit{x} \} \subseteq \{ \mathit{x}, \mathit{y} \}$}
    \RightLabel{\textsc{Extend}}
    \BinaryInfC{$\{ \mathit{x} : \Nat, \mathit{y} : \Singleton{0} \} \sub \{ \mathit{x} : \Int \}$}
    \DisplayProof
  \end{center}
  \caption{Example subtyping derivation for extensible records.}
  \label{fig:subtyping-example}
\end{figure}

Using the \textsc{Trans} rule and the introduction rules, we can derive corresponding elimination rules. (fig. \ref{fig:subtyping-derived})

\begin{figure}[ht]
  % join elim
  \begin{center}
    \AxiomC{$\tau \join \sigma \sub \rho$}
    \RightLabel{\textsc{JoinE}$_1$}
    \UnaryInfC{$\tau \sub \rho$}
    \DisplayProof
    {\hskip.2in}
    \AxiomC{$\tau \join \sigma \sub \rho$}
    \RightLabel{\textsc{JoinE}$_2$}
    \UnaryInfC{$\sigma \sub \rho$}
    \DisplayProof
  \end{center}

  \begin{prooftree}
    \AxiomC{$\tau \sub \sigma \join \rho$}
    \AxiomC{$\sigma \sub \upsilon$}
    \AxiomC{$\rho \sub \upsilon$}
    \RightLabel{\textsc{JoinE}}
    \TrinaryInfC{$\tau \sub \upsilon$}
  \end{prooftree}

  % meet elim
  \begin{center}
    \AxiomC{$\tau \sub \sigma \meet \rho$}
    \RightLabel{\textsc{MeetE}$_1$}
    \UnaryInfC{$\tau \sub \sigma$}
    \DisplayProof
    {\hskip.2in}
    \AxiomC{$\tau \sub \rho \meet \sigma$}
    \RightLabel{\textsc{MeetE}$_2$}
    \UnaryInfC{$\tau \sub \sigma$}
    \DisplayProof
  \end{center}

  \begin{center}
    \AxiomC{$\tau \meet \sigma \sub \rho$}
    \AxiomC{$\sigma \sub \upsilon$}
    \RightLabel{\textsc{MeetE}}
    \BinaryInfC{$\tau \sub \upsilon$}
    \DisplayProof
    {\hskip.2in}
    \AxiomC{$\tau \meet \sigma \sub \rho$}
    \AxiomC{$\tau \sub \upsilon$}
    \RightLabel{\textsc{MeetE}}
    \BinaryInfC{$\sigma \sub \upsilon$}
    \DisplayProof
  \end{center}


  \caption{Derived subtyping rules}
  \label{fig:subtyping-derived}
\end{figure}

% NOTE: If we add a primitive rule between singletons and integers, we can (probably) omit transitivty.


\subsection{Polarity}
\label{sec:polarity}


If we want to use lattice types for type inference (like in section \ref{sec:type-inference}), we have to consider where lattice types can occur.
Generally, union types and the $\bot$ type can only occur in output positions and describe upper bounds.
Intersection types and the $\top$ type on the other hand  can only occur in inout positions and describe lower bounds. \cite{dolan2017subtyping}
The motivation for this can be given as follows: Consider a function that accepts two types as inputs and may return two types as output.
Since the function is assumed to be total any input needs to be of both types (i.e. of the intersection of both types) and the output can be either type (i.e. the union of both types).

\begin{figure}
  \begin{flalign*}
    \tau^+ := & \; \Bool \; | \; \Nat \; | \; \Int \; | \; \String \; | \; \Singleton{0} \; | \; \Singleton{1} \; | \; \Singleton{2} \; | \dots \\
              & \; \bot \; | \; \tau_1^+ \join \tau_2^+ \; | \; \tau_1^- \to \tau_2^+ \; | \; \mu\alpha.\tau^+ \; | \; \alpha                   \\
    \tau^- := & \; \Bool \; | \; \Nat \; | \; \Int \; | \; \String \; | \; \Singleton{0} \; | \; \Singleton{1} \; | \; \Singleton{2} \; | \dots \\
              & \; \top \; | \; \tau_1^- \meet \tau_2^- \; | \; \tau_1^+ \to \tau_2^- \; | \; \mu\alpha.\tau^- \; | \; \alpha
  \end{flalign*}
  \caption{The syntax of polar types}
  \label{fig:polar-type-syntax}
\end{figure}

Therefore, we introduce \emph{polar types} in fig. \ref{fig:polar-type-syntax} as a means to distinguish positive types (for outputs) and negative types (for inputs).
Since functions are covariant in their output and contravariant in their input, the polarity of types in the input position is flipped.
Nominal types can occur both in positive and negative positions.
\todo{clarify role of polar types}

\section{Terms}
\label{sec:terms}

Our language (fig. \ref{fig:term-syntax}) consists of numeric, boolean and String literals corresponding to the nominal types introduced above, as well as the constructs of the lambda calculus:
variables, functions introduced by lambda-abstractions and function applications.
The syntax of terms is extended by record definitions and field selections.

\begin{figure}[ht]
  \begin{flalign*}
    e := & \; x \; | \; y \; | \; z \; | \; \dots                                        & \textit{Variables}                   \\
         & \; \mathit{true} \; | \; \mathit{false}                                       & \textit{Boolean Literals}            \\
         & \; 0 \; | \; 1 \; | \; 2 \; | \; \dots                                        & \textit{Numeric Literals}            \\
         & \; \mathit{"foo"} \; | \; \mathit{"bar"} \; | \; \mathit{"baz"} \; | \; \dots & \textit{String Literals}             \\
         & \; \{ \ell_i = e_1, \dots \} \; | \; e.\ell_i                                 & \textit{Record definition/selection} \\
         & \; \lambda x.e                                                                & \textit{Lambda abstractions}         \\
         & \; e_1e_2                                                                     & \textit{Function applications}
  \end{flalign*}
  \caption{The syntax of terms}
  \label{fig:term-syntax}
\end{figure}