%%%%%%%%%%%%%%%%%%%%%%%%%%%%%%%%%%%%%%%%%%%%%%%%%%%%%%%%%%%%%%%%%%%%
% Diskussion und Ausblick
%%%%%%%%%%%%%%%%%%%%%%%%%%%%%%%%%%%%%%%%%%%%%%%%%%%%%%%%%%%%%%%%%%%%

\chapter{Summary}
\label{ch:summary}

We have shown that not only is it possible to combine qualified types with subtyping, but also that this approach preserves important properties of both concepts.
Covariant predicates are implied for more precise subtypes, contravariant predicates are implied for broader supertypes, while invariant predicates can only be used for semantically equal types.

It remains open, however, how generic programming works in conjunction with this approach.
Further work hast to be done in correctly combining the biunification algorithm of subtyping constraints and the minimisation in type automata with type class constraints.

% Further work: Type simplification with type class constraints, type inference, applications for different subtyping systems

\clearpage
