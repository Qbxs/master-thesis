
\chapter{Subtyping}\label{ch:subtyping}

\section{Motivation}\label{sec:subtypemotivation}

In many cases the specific semantics of types exhibit a hierarchy.
In Object-oriented programming this hierarchy is given in the form of sub- and superclasses.
We can express the relationship between super- and subclasses, or more generally super- and subtypes,
in the form that all properties of the superclass is also exhibited in the subclass. \cite{subtyping}

In the case of OO-languages this means that, if the class \texttt{SubC} is a subclass of \texttt{SuperC},
then any method defined in \texttt{SuperC} is also going to be defined for objects of \texttt{SubC}.
This enables us to use an object \texttt{SubC} wherever a \texttt{SuperC} is expected.

We denote $T \leq S$ for $T$ is a subtype of $S$.
Syntactically, this implies that if we have obatined the judgement $e : T$, we also have $e : S$.
Therefore, we can use $e$ in any context that expects the usage of a term of type $S$.
Semantically, the subtyping relation can be understood analogously to sets in terms of the subset relationship $\subseteq$,
meaning all terms $e$ of type $T$ are also of type $S$.
\cite{reynolds_1998}

This permits many useful features in programming languages such as the reuse and abstraction of code to the supertypes and implicit coercions from a subtype to a supertype.
