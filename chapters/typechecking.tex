\chapter{Type System}
\label{ch:type-system}
% explain type inference with algebraic subtyping/bounds/bisubstitutions

\section{Typing Rules}

\begin{figure}[h]
\begin{center}
\AxiomC{}
\RightLabel{\textsc{T-Var}}
\UnaryInfC{$\Gamma, x : \tau \vdash x : \tau$}
\DisplayProof
{\hskip.2in}
\AxiomC{$\ctx e_1 : \sigma \to \tau$}
\AxiomC{$\ctx e_2 : \sigma$}
\RightLabel{\textsc{T-App}}
\BinaryInfC{$\ctx e_1 \; e_2 : \tau$}
\DisplayProof

\AxiomC{$\Gamma, x : \sigma \vdash e : \tau$}
\RightLabel{\textsc{T-Abs}}
\UnaryInfC{$\ctx \lambda x.e : \sigma \to \tau$}
\DisplayProof
{\hskip.2in}
\AxiomC{$\ctx e : \sigma$}
\AxiomC{$\sigma \sub \tau$}
\RightLabel{\textsc{T-Sub}}
\BinaryInfC{$\ctx e : \tau$}
\DisplayProof
\end{center}
\caption{Typing Rules}
\label{fig:typing-rules}
\end{figure}

The resolution rules for type class witnesses in section \ref{sec:witnesses} provide us with means to implement type checking for type class methods.
For every call to a type class method resolution either fails or provides a dictionary, i.e. a witness, for the type class constraint.

What makes this kind of typing derivation special is that it not only decomposes judgements upwards,
but fills in the gap displayed by the type class constraint downwards by resolving a fitting dictionary.

Type inference for type class method calls (fig. \ref{fig:showable-example}):
$w$ is the implicit witness for the type class constraint $\Showable(\tau)$ which is inferred by type class resolution.
$k$ is the continuation that is passed to the type class method.

Generally we generate constraints between the types introduced in the class declaration of the method and the unification variables generated for its arguments.
In this case $\mathbf{show}$ is implemented in continuation passing style, so apart from the argument to be shown $x$, there is continutation $k$ which consumes the $\mathit{String}$ output.
The argument $w$ is implicit and stands for the inferred witness of the type class.
At compilation it will be filled with a resolved instance, so that the evaluation of the metod is well defined.

\begin{figure}[h]
    \centering
    \AxiomC{$\tau$ fresh}
    \AxiomC{$\ctx w : \Showable(\tau)$}
    \AxiomC{$\ctx x : \tau$}
    \AxiomC{$\ctx k : \String \to \sigma$}
    \RightLabel{\textsc{T-Class}$^+$}
    \QuaternaryInfC{$\ctx \showTerm w \; x \; k : \sigma$}
    \DisplayProof
\caption{Derivation for $\showTerm w \; x \; k : \sigma$}
\label{fig:showable-example}
\end{figure}

For contravariant type classes such as $\Readable$ the inference looks slightly different:
In this case, the type class constraint is applied to the argument type of the continuation.

\begin{figure}[h]
    \centering
    \AxiomC{$\tau$ fresh}
    \AxiomC{$\ctx w : \Readable(\tau)$}
    \AxiomC{$\ctx x : \String$}
    \AxiomC{$\ctx k : \tau \to \sigma$}
    \RightLabel{\textsc{T-Class}$^-$}
    \QuaternaryInfC{$\ctx \readTerm w \; x \; k : \sigma$}
    \DisplayProof
\caption{Derivation for $\readTerm w \; x \; k : \sigma$}
\label{fig:readable-example}
\end{figure}


Outlined below is the algorithm used to infer a principle(?) type for any given term without the presence of type classes.
We will then examine, which alterations need to be done in order to achieve the same with the addition of type class constraints.

\section{Type Inference}
\label{sec:type-inference}

The core of type inference is done using an unifaction algorithm.
For this, in the first step, we generate subtyping and type class constraint based on the term.
When solving the constraint set, the algorithm either fails (i.e. the term does not type check) or for all unification variables we obtain lower and upper bounds, as well as type class constraints.
The resulting bisubstitution is used to compute a principal type which can later be simplified using a translation to its type automata representation which is being minimized.
As a result, we obtain a simple principal type for any term that typechecks.

\subsection{Constraint Generation}


\begin{figure}[h]
\begin{center}
    $\constraintGen{\alpha}_\tau^\bot := \alpha$ \\
    $\constraintGen{\alpha}_\tau^\top := \alpha$ \\
    $\constraintGen{\nu \to \nu'}_\tau^\bot := \constraintGen{\nu}_\tau^\top \to \constraintGen{\nu'}_\tau^\bot$ \\
    $\constraintGen{\nu \to \nu'}_\tau^\top := \constraintGen{\nu}_\tau^\bot \to \constraintGen{\nu'}_\tau^\top$ \\
    $\constraintGen{\mu\alpha.\nu}_\tau^\bot := \mu\alpha.\constraintGen{\nu}_\tau^\bot$ \\
    $\constraintGen{\mu\alpha.\nu}_\tau^\top := \mu\alpha.\constraintGen{\nu}_\tau^\top$ \\
\end{center}
\caption{Upper/lower bound translation}
\label{fig:bound-translation}
\end{figure}

\begin{figure}[h]
    \begin{center}

        \AxiomC{$\Gamma(x) = \tau$}
        \RightLabel{\textsc{Var}}
        \UnaryInfC{$\Gamma \vdash x : \tau \toConstr \emptyset$}
        \DisplayProof

        \AxiomC{$\Gamma \vdash e : \tau \toConstr \Xi$}
        \AxiomC{$\mathit{Fresh}(\beta)$}
        \RightLabel{\textsc{Lam}}
        \BinaryInfC{$\Gamma \vdash \lambda x. e : \beta \to \tau \toConstr \Xi$}
        \DisplayProof

        \AxiomC{$\Gamma \vdash e_1 : \sigma_1 \toConstr \Xi_1$}
        \AxiomC{$\Gamma \vdash e_2 : \sigma_2 \toConstr \Xi_2$}
        \AxiomC{$\mathit{Fresh}(\beta)$}
        \RightLabel{\textsc{App}}
        \TrinaryInfC{$\Gamma \vdash e_1e_2 : \beta \toConstr \{ \sigma_1 \sub \sigma_2 \to \beta \} \cup \Xi_1 \cup \Xi_2$}
        \DisplayProof

        \AxiomC{$\Gamma \vdash x : \sigma \toConstr \Xi$}
        \AxiomC{$P(\mathbf{m}) := \forall \alpha. \Phi(\alpha) \Rightarrow \alpha \to \rho$}
        \AxiomC{$\mathit{Fresh}(\beta)$}
        \RightLabel{\textsc{Method}$^+$}
        \TrinaryInfC{$\Gamma \vdash \mathbf{m} \; x : \tau \toConstr \{ \Phi_m(\beta), \sigma \sub \beta, \tau \sub \rho \} \cup \Xi$}
        \DisplayProof

        \AxiomC{$\Gamma \vdash x : \sigma \toConstr \Xi$}
        \AxiomC{$P(\mathbf{m}) := \forall \alpha. \Phi(\alpha) \Rightarrow \rho \to \alpha$}
        \AxiomC{$\mathit{Fresh}(\beta)$}
        \RightLabel{\textsc{Method}$^-$}
        \TrinaryInfC{$\Gamma \vdash \mathbf{m} \; x : \tau \toConstr \{ \Phi_m(\beta), \beta \sub \tau, \sigma \sub \rho \} \cup \Xi$}
        \DisplayProof

    \end{center}
    \caption{Constraint Generation}
    \label{fig:constraint-generation}
\end{figure}

\subsection{Constraint Solving}

\begin{figure}[h]
    \begin{center}
        placeholder: rules for solvinf constraints
        \begin{itemize}
            \item cache hits
            \item adding to lower/upper bounds
            \item decomposing
            \item fail
        \end{itemize}
    \end{center}
    \caption{Constraint Solving}
    \label{fig:constraint-solving}
\end{figure}

\begin{figure}[h]
    % We might add witnesses here?
    \begin{flalign*}
        \decompose{\tau}{\top} & := & \emptyset \\
        \decompose{\bot}{\tau} & := & \emptyset \\
        \decompose{\tau_1 \join \tau_2}{\sigma} & := & \{ \tau_1 \sub \sigma, \tau_2 \sub \sigma \} \\
        \decompose{\sigma}{\tau_1 \meet \tau_2} & := & \{ \sigma \sub \tau_1, \sigma \sub \tau_2 \} \\
        \decompose{\tau}{\mu\alpha.\sigma} & := & \{ \tau \sub \sigma[\mu\alpha.\sigma / \alpha] \} \\
        \decompose{\mu\alpha.\sigma}{\tau} & := & \{ \sigma[\mu\alpha.\sigma / \alpha] \sub \tau \} \\
        \decompose{\alpha}{\alpha} & := & \emptyset \\
        \decompose{\sigma_1 \to \tau_1}{\sigma_2 \to \tau_2} & := & \{ \sigma_2 \sub \sigma_1, \tau_1 \sub \tau_2 \} \\
    \end{flalign*}
    \caption{Constraint Decomposition}
    \label{fig:constraint-decomposition}
\end{figure}

\subsection{Type Coalescing}

\subsection{Type Simplification}