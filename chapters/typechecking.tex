\chapter{Type Inference}\label{ch:type-inference}
% explain type inference with algebraic subtyping/bounds/bisubstitutions

\section{Bisubstitution}

For type inference of type class methods we have to consider the appropiate bounds for the unification variable in question.
I.e. for covariant type classes we need to be able to resolve an instance for the upper bounds,
and for covariant type classes we need to resolve an instance for the lower bounds.

\section{Typing Rules}

\begin{figure}[h]
\begin{center}
\AxiomC{}
\RightLabel{\textsc{T-Var}}
\UnaryInfC{$\Gamma, x : \tau \vdash x : \tau$}
\DisplayProof
{\hskip.2in}
\AxiomC{$\Gamma \vdash e_1 : \sigma \to \tau$}
\AxiomC{$\Gamma \vdash e_2 : \sigma$}
\RightLabel{\textsc{T-App}}
\BinaryInfC{$\Gamma \vdash e_1 \; e_2 : \tau$}
\DisplayProof

\AxiomC{$\Gamma x : \sigma \vdash e : \tau$}
\RightLabel{\textsc{T-Abs}}
\UnaryInfC{$\Gamma \vdash \lambda x.e : \sigma \to \tau$}
\DisplayProof
{\hskip.2in}
\AxiomC{$\Gamma \vdash e : \sigma$}
\AxiomC{$\sigma \sub \tau$}
\RightLabel{\textsc{T-Sub}}
\BinaryInfC{$\Gamma \vdash e : \tau$}
\DisplayProof
\end{center}
\caption{Typing Rules}
\label{fig:typing-rules}
\end{figure}
