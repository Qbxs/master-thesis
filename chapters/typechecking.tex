\chapter{Type Inference}\label{ch:type-inference}
% explain type inference with algebraic subtyping/bounds/bisubstitutions

\section{Bisubstitution}

For type inference of type class methods we have to consider the appropiate bounds for the unification variable in question.
I.e. for covariant type classes we need to be able to resolve an instance for the upper bounds,
and for covariant type classes we need to resolve an instance for the lower bounds.

\section{Typing Rules}

\begin{figure}[h]
\begin{center}
\AxiomC{}
\RightLabel{\textsc{T-Var}}
\UnaryInfC{$\Gamma, x : \tau \vdash x : \tau$}
\DisplayProof
{\hskip.2in}
\AxiomC{$\ctx e_1 : \sigma \to \tau$}
\AxiomC{$\ctx e_2 : \sigma$}
\RightLabel{\textsc{T-App}}
\BinaryInfC{$\ctx e_1 \; e_2 : \tau$}
\DisplayProof

\AxiomC{$\Gamma, x : \sigma \vdash e : \tau$}
\RightLabel{\textsc{T-Abs}}
\UnaryInfC{$\ctx \lambda x.e : \sigma \to \tau$}
\DisplayProof
{\hskip.2in}
\AxiomC{$\ctx e : \sigma$}
\AxiomC{$\sigma \sub \tau$}
\RightLabel{\textsc{T-Sub}}
\BinaryInfC{$\ctx e : \tau$}
\DisplayProof
\end{center}
\caption{Typing Rules}
\label{fig:typing-rules}
\end{figure}

Type inference for type class method calls (fig. \ref{fig:showable-example}):
$w$ is the implicit witness for the type class constraint $\Showable(\tau)$ which is inferred by type class resolution.
$k$ is the continuation that is passed to the type class method.

\begin{figure}[h]
    \centering
    \AxiomC{$\ctx w : \Showable(\tau)$}
    \AxiomC{$\ctx x : \tau$}
    \AxiomC{$\ctx k : \String \to \sigma$}
    \RightLabel{\textsc{T-Class}}
    \TrinaryInfC{$\ctx \showTerm w \; x \; k : \sigma$}
    \DisplayProof
\caption{Derivation for $\showTerm w \; x \; k : \sigma$}
\label{fig:showable-example}
\end{figure}