\chapter{Type Classes}\label{ch:typeclasses}

In order to generalize functions over data types there have been several proposals to abstract over types in different programming languages:
Amongst others,
\begin{itemize}
\item Parametric polymorphism: we can define functions without knowing the concrete representation of the arguments and result types (e.g. \texttt{id : a -> a}) % cite something
\item ad-hoc polymorphism: in some mostly imperative languages it is possible to simply overload functions (e.g. we may define \texttt{+} both on integers and on float values, the correct implementation is then picked based on the argument type) % cite...
\item and type classes. Combining both approaches (making ad-polymorphism less ad-hoc) we can define classes and corresponding functions and then use them to constrain valid argument types when defining a function that relies on these type class functions \cite{wadlerblott}
\end{itemize}

\section{Type Classes}
% overloading vs type classes
% paper by wadler

If our type system doesn't allow for ad-hoc polymorphism it may seem neccessary to write verbose code for basic function with respect to every concrete type it should be used for.
An intuitive example for this (that is also the motivation for type classes in the original proposal) are arithmetic operators.
We simply cannot define \texttt{(+) : Int -> Int -> Int} and then also \texttt{(+) : Float -> Float -> Float} in a different implementation, on which both types definetly rely on under the hood.
But these types have nonetheless something in common. Namely that they both stand for \emph{numerical} values, that hence support the usual arithmetic operations, like addition, multiplication, division and so on.

The idea of type classes is to generalise attributes of types with appropriate function.
The class \mintinline{Haskell}|Num| in Haskell expects that we can implement a number of numerical functions for a type $\tau$ if it is ought to be a member of the \mintinline{Haskell}|Num| type class.

Shortened definition of the \mintinline{Haskell}|Num| type class in Haskell.
% \footnote[1]{As can be found in the default \texttt{Prelude}: \url{https://hackage.haskell.org/package/base-4.16.2.0/docs/Prelude.html#t:Num}}

\begin{minted}{Haskell}
class  Num a  where
    (+), (-), (*)       :: a -> a -> a
\end{minted}

An instance would look like:

\begin{minted}{Haskell}
instance  Num Int  where
    (+) = intAdd
    (-) = intMinus
    (*) = intMul
\end{minted}





% later:

In the end, this enables us to use elaborate concepts such as functors and monads to reason about programs.

\cite{wadlerblott}

\section{The problem of type class coherence}
% for each type there t may be globally at most one instance C t defined
% can already be tricky in a modular system
% more challenging with subtyping

Uniqueness or non-ambiguity of type classes.
