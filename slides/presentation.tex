\documentclass{beamer}
\usepackage[utf8]{inputenc}
\usepackage{cite}
\usepackage[english]{babel}
\usepackage{amsmath}
\usepackage{amsthm}
\usepackage{stmaryrd}
\usepackage{minted}
\usepackage{wasysym}
\usepackage{tikz-cd}
\usepackage{adjustbox}
\usepackage{bbold}

\usetikzlibrary{babel}

%\usepackage{reportpage}
\usepackage{amsmath}
\usepackage{bbold}
\usepackage{amssymb}
\usepackage{stmaryrd}
\usepackage{mathrsfs}
\usepackage{scalerel}
\usepackage{multirow}
\usepackage{epsf}
\usepackage{graphics, graphicx}
\usepackage{latexsym}
\usepackage[margin=10pt,font=small,labelfont=bf]{caption}
\usepackage[utf8]{inputenc}
\usepackage[toc,page]{appendix}
\usepackage{subcaption}
\usepackage{tikz}
\usepackage{tikz-cd}
\tikzcdset{scale cd/.style={every label/.append style={scale=#1},
    cells={nodes={scale=#1}}}}
\usepackage{xcolor}
\usetikzlibrary{cd}

\usepackage{amsthm}

\usepackage{lineno}
% \linenumbers

\usepackage{listings}
% basic duo style
\lstdefinestyle{duoStyle}{%
  otherkeywords={case,cocase,def,of,forall,data,codata,rec,mu,import,module,return,set,Top,Bot,CBV,CBN,refinement,constructor,destructor,type,operator,leftassoc,rightassoc,cmd,prd,cns,class,instance,:=,=>,=<,>>,\\\\/,/\\\\,<:,\[,\]},
  morekeywords=[1]{case,cocase,def,of,forall,data,codata,rec,mu,import,module,return,set,Top,Bot,CBV,CBN,refinement,constructor,destructor,type,operator,leftassoc,rightassoc,cmd,prd,cns,class,instance},
  morekeywords=[2]{:=,=>,=<,>>,\\\\/,/\\\\,<:,\[,\]},
  keywordstyle=[1]{\bf\color{blue}},
  keywordstyle=[2]{\bf\color{black}},
  comment=[l]{--},
  commentstyle=\color{gray},
  keywordstyle=\ttfamily\bfseries\color{blue!80!black},
  mathescape=true
  %moredelim=**[is][\bfseries]{|}{|}
}
\lstset{style=duoStyle}


\newenvironment{code}{\captionsetup{type=listing, position=bottom}}{}

% mathbb String
\usepackage{bussproofs}
\EnableBpAbbreviations
\def\ScoreOverhang{0pt}
\def\buildNoScoreTight{% Make an hbox with no score
 \global\setbox\myBoxD=\hbox{\vbox{\vskip0pt}}% 0pt instead of 1pt
}
\def\noLine{% noLine with less space between lines
 \gdef\buildScore{\buildNoScoreTight}%
 \ignorespaces%
}

% noindent on new paragraph
\setlength\parindent{0pt}

% Inference rules
\newcommand{\subPosRule}{\RightLabel{\textsc{Sub}$^+$}}
\newcommand{\subNegRule}{\RightLabel{\textsc{Sub}$^-$}}
\newcommand{\instanceDeclRule}{\RightLabel{\textsc{Decl}}}
\newcommand{\meetRule}{\RightLabel{\textsc{Meet}}}
\newcommand{\joinRule}{\RightLabel{\textsc{Join}}}
\newcommand{\axiomPos}{\RightLabel{\textsc{Axiom}$^+$}}
\newcommand{\axiomNeg}{\RightLabel{\textsc{Axiom}$^-$}}
\newcommand{\topRule}{\RightLabel{\textsc{Top}}}
\newcommand{\botRule}{\RightLabel{\textsc{Bot}}}

\newcommand\deriveRule{
  {\hskip.1in}
  $\stackrel{\mathit{drv}}{\rightsquigarrow}$
  {\hskip.1in}}

\newcommand{\ctx}{\ensuremath{\Gamma \; \vdash \;}}
\newcommand{\opeq}{\ensuremath{\stackrel{op}{\simeq}}}

% Symbols
\newcommand\join{\vee}
\newcommand\meet{\wedge}
\newcommand\sub{\ensuremath{\leqslant}}
\newcommand\transition[1]{\ensuremath{\xrightarrow[]{#1}}}

% Text
\newcommand\Showable{\ensuremath{\Psi_\mathit{Show}}}
\newcommand\Readable{\ensuremath{\Psi_\mathit{Read}}}
\newcommand\Defaultable{\ensuremath{\Psi_\mathit{Default}}}
\newcommand\Ordable{\ensuremath{\Psi_\mathit{Ord}}}
\newcommand\Eq{\ensuremath{\Psi_\mathit{Eq}}}
\newcommand\Monoid{\ensuremath{\Psi_\mathit{Monoid}}}
\newcommand\instance[2]{{\bf instance} \; #1(#2)}
\newcommand\class[2]{{\bf class} \; #1(#2)}

% Terms
\newcommand\caseof[1]{{\bf case}\; #1 \;{\bf of}}
\newcommand\showTerm{{\bf show}}
\newcommand\readTerm{{\bf read}}
\newcommand\ifthenelseTerm{{\bf ifthenelse}\;}
\newcommand\eq{{\bf eq}\;}

% Types
\newcommand\Nat{\mathbb{N}}
\newcommand\Int{\mathbb{Z}}
\newcommand\Unit{\mathbb{U}}
\newcommand\Bool{\mathbb{B}}
\newcommand\Singleton[1]{\mathbb{1}_#1}
\newcommand\String{\mathbb{S}}

% Witnesses
\newcommand\cov{\mathit{cov}}
\newcommand\contrav{\mathit{contrav}}
\newcommand\inv{\mathit{inv}}
% relational composition operator
\newcommand*{\comp}{\mathbin{\raise 0.6ex\hbox{\oalign{\hfil$\scriptscriptstyle
            		 \rm o$\hfil\cr\hfil$\scriptscriptstyle\rm 9$\hfil}}}}
\newcommand\refl{\mathit{refl}}
\newcommand\topW{\mathit{top}}
\newcommand\natPrim{\mathit{natPrim}}
\newcommand\botW{\mathit{bot}}
\newcommand\meetW{\mathit{meet}}
\newcommand\joinW{\mathit{join}}
\newcommand\inW[1]{\mathit{in}_{#1}}
\newcommand\proj[1]{\mathit{proj}_{#1}}
\newcommand\funcW{\mathit{func}}
\newcommand\unfoldL{\mathit{unfold}_{L}}
\newcommand\unfoldR{\mathit{unfold}_{R}}

\newcommand\constraintGen[1]{\ensuremath{\llbracket #1 \rrbracket}}
\newcommand\toConstr{\ensuremath{\hookrightarrow}}
\newcommand\decompose[2]{\ensuremath{{\mathbf{decompose}}(#1 \sub #2)}}

\usepackage{mathpartir}

\setminted{fontsize=\footnotesize}

\title{Type Class Coherence with Subtyping}
\author{Pascal Engel}
\date{June 22 2023}

\begin{document}

\begin{frame}
    \maketitle
\end{frame}

\begin{frame}
    \frametitle{Structure}
    \tableofcontents
\end{frame}

\section{Introduction}

\begin{frame}
    \frametitle{Introduction}

    \begin{itemize}
        \item Existing forms of polymorphism to achieve different kinds of program abstraction: Parametricity, subtyping and overloading
        \item New: Combining type classes with subtyping
    \end{itemize}
\end{frame}

\begin{frame}
  \frametitle{Introduction}

  $\Phi_{\mathit{Read}}(\{ x : \Nat, y : \Int \}) \Rightarrow  \Phi_{\mathit{Read}}(\{ x : \Nat \})$
\end{frame}

\section{Polymorphism}

\begin{frame}
    \frametitle{Polymorphism}

    \begin{itemize}
      \item Parametricity: Theorems for free
      \item Subtyping: Liskov substitution
      \item Type classes: Coherent overloading/resolution
    \end{itemize}
\end{frame}

\section{The Lattice of Types}

\begin{frame}
    \frametitle{The Lattice of Types}

    \begin{prooftree}
        \AxiomC{}
        \RightLabel{\textsc{Refl}}
        \UnaryInfC{$\tau \sub \tau$}
      \end{prooftree}
    
      % trans
      \begin{prooftree}
        \AxiomC{$\tau \sub \sigma$}
        \AxiomC{$\sigma \sub \rho$}
        \RightLabel{\textsc{Trans}}
        \BinaryInfC{$\tau \sub \rho$}
      \end{prooftree}
    
      % top/bot
      \begin{center}
        \AxiomC{}
        \RightLabel{\textsc{Top}}
        \UnaryInfC{$\tau \sub \top$}
        \DisplayProof
        {\hskip.2in}
        \AxiomC{}
        \RightLabel{\textsc{Bot}}
        \UnaryInfC{$\bot \sub \tau$}
        \DisplayProof
      \end{center}
    
      % join intro
      \begin{center}
        \AxiomC{$\tau \sub \sigma$}
        \RightLabel{\textsc{JoinI}$_1$}
        \UnaryInfC{$\tau \sub \sigma \join \rho$}
        \DisplayProof
        {\hskip.2in}
        \AxiomC{$\tau \sub \sigma$}
        \RightLabel{\textsc{JoinI}$_2$}
        \UnaryInfC{$\tau \sub \rho \join \sigma$}
        \DisplayProof
      \end{center}
    
      \begin{prooftree}
        \AxiomC{$\tau \sub \rho$}
        \AxiomC{$\sigma \sub \rho$}
        \RightLabel{\textsc{JoinI}}
        \BinaryInfC{$\tau \join \sigma \sub \rho$}
      \end{prooftree}
    
      % meet intro
      \begin{center}
        \AxiomC{$\tau \sub \rho$}
        \RightLabel{\textsc{MeetI}$_1$}
        \UnaryInfC{$\tau \meet \sigma \sub \rho$}
        \DisplayProof
        {\hskip.2in}
        \AxiomC{$\sigma \sub \rho$}
        \RightLabel{\textsc{MeetI}$_2$}
        \UnaryInfC{$\tau \meet \sigma \sub \rho$}
        \DisplayProof
      \end{center}
    
      \begin{prooftree}
        \AxiomC{$\tau \sub \rho$}
        \AxiomC{$\tau \sub \sigma$}
        \RightLabel{\textsc{MeetI}}
        \BinaryInfC{$\tau \sub \sigma \meet \rho$}
      \end{prooftree}
    
      \begin{prooftree}
        \AxiomC{$\tau' \sub \tau$}
        \AxiomC{$\sigma \sub \sigma'$}
        \RightLabel{\textsc{Func}}
        \BinaryInfC{$\tau \to \sigma \sub \tau' \to \sigma'$}
      \end{prooftree}
    
      \begin{prooftree}
        \AxiomC{$\forall j. \tau_j \sub \sigma_j$}
        \AxiomC{$\{ \overline{\ell_j} \} \subseteq \{ \overline{\ell_i} \}$}
        \RightLabel{\textsc{Extend}}
        \BinaryInfC{$\{ \overline{\ell_i : \tau_i} \} \sub \{ \overline{\ell_j : \sigma_j} \}$}
      \end{prooftree}
    
      % recursive types
      \begin{center}
        \AxiomC{}
        \RightLabel{\textsc{Rec}$_1$}
        \UnaryInfC{$\mu\rho.\tau \sub \tau [\mu\rho.\tau\ / \rho]$}
        \DisplayProof
        {\hskip.2in}
        \AxiomC{}
        \RightLabel{\textsc{Rec}$_2$}
        \UnaryInfC{$\tau[\mu\rho.\tau / \rho] \sub \mu\rho.\tau$}
        \DisplayProof
      \end{center}
\end{frame}

\section{Predicates on Types}

\begin{frame}
    \frametitle{Predicates on Types}
    A subset of propositions about types that is transferable via the subtyping lattice.
    \begin{prooftree}
      \AxiomC{$\ctx \Phi^<(\sigma)$}
      \AxiomC{$\tau \sub \sigma$}
      \alwaysSingleLine
      \RightLabel{\textsc{CoV}}
      \BinaryInfC{$\ctx \Phi^<(\tau)$}
    \end{prooftree}

    \begin{prooftree}
      \alwaysNoLine
      \AxiomC{$\ctx \Phi^>(\sigma)$}
      \AxiomC{$\sigma \sub \tau$}
      \alwaysSingleLine
      \RightLabel{\textsc{ContraV}}
      \BinaryInfC{$\ctx \Phi^>(\tau)$}
    \end{prooftree}
\end{frame}

\section{Predicates on Types}

\begin{frame}
    \frametitle{Predicates on Types}
    Examples of co-/contravariant type classes (Show, Eq, Ord, Read, Default)
\end{frame}

\section{Type Class Coherence}

\begin{frame}
    \frametitle{Type Class Coherence}

    Resolution can't go wrong, evidence is to a certain degree unique (behavioral equality)
\end{frame}

\begin{frame}
  \frametitle{Type Class Coherence}

%   \begin{tikzcd}
%     &                                                      & {\color{blue}\{\}} \arrow[lld] \arrow[ld] \arrow[d] \arrow[rrd]   &       &                                     \\
%     \{ \ell_1 : \tau_1 \} \arrow[d] \arrow[rrd]           & {\color{blue}\{ \ell_2 : \tau_2 \}} \arrow[d] \arrow[ld]           & {\color{blue}\{ \ell_3 : \tau_3 \}} \arrow[d] \arrow[ld]          & \dots & \{ \ell_n : \tau_n \} \arrow[llddd] \\
%     {\{ \ell_1 : \tau_1, \ell_2 : \tau_2 \}} \arrow[rrdd] & {\color{blue}\{ \ell_2 : \tau_2, \ell_3 : \tau_3 \}} \arrow[rdd] & {\{ \ell_1 : \tau_1, \ell_3 : \tau_3 \}} \arrow[dd] & \dots &                                     \\
%     &                                                      & \dots                                               &       &                                     \\
%     &                                                      & {\{ \ell_1 : \tau_1, \dots \ell_n : \tau_n \}}      &       &
%   \end{tikzcd}

\end{frame}

\begin{frame}
    \frametitle{Examples}
\end{frame}

\begin{frame}
    \frametitle{Summary and Future Work}
    Type simplification, emptiness check

    Example ref
    \cite{kiselyov}
\end{frame}

\begin{frame}
    \frametitle{Bibliography}
    \bibliographystyle{alpha}
    \bibliography{../literature}
\end{frame}

\end{document}