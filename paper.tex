%%%%%%%%%%%%%%%%%%%%%%%%%%%%%%%%%%%%%%%%%%%%%%%%%%%%%%%%%%%%%%%%%%%%%%%%%%%%%
%%% LaTeX-Rahmen fuer das Erstellen von Masterarbeiten
%%%%%%%%%%%%%%%%%%%%%%%%%%%%%%%%%%%%%%%%%%%%%%%%%%%%%%%%%%%%%%%%%%%%%%%%%%%%%

%%%%%%%%%%%%%%%%%%%%%%%%%%%%%%%%%%%%%%%%%%%%%%%%%%%%%%%%%%%%%%%%%%%%%%%%%%%%%
%%% allgemeine Einstellungen
%%%%%%%%%%%%%%%%%%%%%%%%%%%%%%%%%%%%%%%%%%%%%%%%%%%%%%%%%%%%%%%%%%%%%%%%%%%%%

\documentclass[twoside,12pt,a4paper]{report}
%\usepackage{reportpage}
\usepackage{amsmath}
\usepackage{scalerel}
\usepackage{epsf}
\usepackage{graphics, graphicx}
\usepackage{latexsym}
\usepackage[margin=10pt,font=small,labelfont=bf]{caption}
\usepackage[utf8]{inputenc}
\usepackage[toc,page]{appendix}
\usepackage{tikz-cd}
\usetikzlibrary{cd}

\usepackage{amsthm}
\newtheorem{definition}{Definition}

\usepackage[hyphens]{url} % 'hyphens' option allows line breaks after "-" characters
\usepackage[ colorlinks = true
           , linkcolor = black
           , anchorcolor = black
           , citecolor = black
           , urlcolor = blue]{hyperref}

\usepackage{listings}
% basic duo style
\lstdefinestyle{duoStyle}{%
    morekeywords={if, then, else instance, class, >>, CBV, CBN, +, -, =>, case, of, cocase, mu, def, prd, cns, cmd, :=},
    keywordstyle=\ttfamily\bfseries\color{blue!80!black},
    %numbers=none, % do not uncomment! issues when used inside arrays!
    mathescape=true
    %moredelim=**[is][\bfseries]{|}{|}
}
\lstset{style=duoStyle}

\usepackage{minted}


\usepackage{bussproofs}
\EnableBpAbbreviations
\def\ScoreOverhang{0pt}
\def\buildNoScoreTight{% Make an hbox with no score
 \global\setbox\myBoxD=\hbox{\vbox{\vskip0pt}}% 0pt instead of 1pt
}
\def\noLine{% noLine with less space between lines
 \gdef\buildScore{\buildNoScoreTight}%
 \ignorespaces%
}

% Inference rules
\newcommand{\subPosRule}{\RightLabel{\textsc{Sub}$^+$}}
\newcommand{\subNegRule}{\RightLabel{\textsc{Sub}$^-$}}
\newcommand{\instanceDeclRule}{\RightLabel{\textsc{Decl}}}
\newcommand{\meetRule}{\RightLabel{\textsc{Meet}}}
\newcommand{\joinRule}{\RightLabel{\textsc{Join}}}
\newcommand{\axiomPos}{\RightLabel{\textsc{Axiom}$^+$}}
\newcommand{\axiomNeg}{\RightLabel{\textsc{Axiom}$^-$}}
\newcommand{\topRule}{\RightLabel{\textsc{Top}}}
\newcommand{\botRule}{\RightLabel{\textsc{Bot}}}

\newcommand{\ctx}{\ensuremath{\Gamma \; \vdash \;}}

% Meets and Joins
\newcommand\join{\scalerel*{\bigvee}{i}}
\newcommand\meet{\scalerel*{\bigwedge}{i}}

\usepackage{mathpartir}

\textwidth 14cm
\textheight 22cm
\topmargin 0.0cm
\evensidemargin 1cm
\oddsidemargin 1cm
%\footskip 2cm
\parskip0.5explus0.1exminus0.1ex

% Kann von Student auch nach pers\"onlichem Geschmack ver\"andert werden.
\pagestyle{headings}

\sloppy

% not working as intended
% \newcommand{\textHaskell}{1}{\mint{Haskell}|$1|}

\begin{document}

%%%%%%%%%%%%%%%%%%%%%%%%%%%%%%%%%%%%%%%%%%%%%%%%%%%%%%%%%%%%%%%%%%%%%%%%%%%%
%%% hier steht die neue Titelseite 
%%%%%%%%%%%%%%%%%%%%%%%%%%%%%%%%%%%%%%%%%%%%%%%%%%%%%%%%%%%%%%%%%%%%%%%%%%%%
 
\begin{titlepage}
 \begin{center}
  {\LARGE Eberhard Karls Universit\"at T\"ubingen}\\
  {\large Mathematisch-Naturwissenschaftliche Fakult\"at \\
Wilhelm-Schickard-Institut f\"ur Informatik\\[4cm]}
  {\huge Master Thesis Informatics\\[2cm]}
  {\Large\bf  Type Class Coherence with Subtyping\\[1.5cm]}
 {\large Pascal Engel}\\[0.5cm]
Datum\\[4cm]
{\small\bf Reviewers}\\[0.5cm]
  \parbox{7cm}{\begin{center}{\large Name Erstgutachter}\\
   (Informatik)\\
  {\footnotesize Wilhelm-Schickard-Institut f\"ur Informatik\\
	Universit\"at T\"ubingen}\end{center}}\hfill\parbox{7cm}{\begin{center}
  {\large Name Zweitgutachter}\\
  (Informatik)\\
  {\footnotesize Wilhelm-Schickard-Institut f\"ur Informatik\\
	Universit\"at T\"ubingen}\end{center}
 }
  \end{center}
\end{titlepage}

%%%%%%%%%%%%%%%%%%%%%%%%%%%%%%%%%%%%%%%%%%%%%%%%%%%%%%%%%%%%%%%%%%%%%%%%%%%%
%%% Titelr"uckseite: Bibliographische Angaben
%%%%%%%%%%%%%%%%%%%%%%%%%%%%%%%%%%%%%%%%%%%%%%%%%%%%%%%%%%%%%%%%%%%%%%%%%%%%

\thispagestyle{empty}
\vspace*{\fill}
\begin{minipage}{11.2cm}
\textbf{Engel, Pascal:}\\
\emph{Type Class Coherence with Subtyping}\\ Master Thesis Informatics\\
Eberhard Karls Universit\"at T\"ubingen\\
Thesis period: von-bis
\end{minipage}
\newpage

%%%%%%%%%%%%%%%%%%%%%%%%%%%%%%%%%%%%%%%%%%%%%%%%%%%%%%%%%%%%%%%%%%%%%%%%%%%%

\pagenumbering{roman}
\setcounter{page}{1}

%%%%%%%%%%%%%%%%%%%%%%%%%%%%%%%%%%%%%%%%%%%%%%%%%%%%%%%%%%%%%%%%%%%%%%%%%%%%
%%% Seite I: Zusammenfassug, Danksagung
%%%%%%%%%%%%%%%%%%%%%%%%%%%%%%%%%%%%%%%%%%%%%%%%%%%%%%%%%%%%%%%%%%%%%%%%%%%%


\section*{Abstract}

In this work we aim to bring together three notions of polymorphism:
Subtyping as a form of ad hoc polymorphism, parametric polymorphism and type classes.
We do so by extending the research language \texttt{duo}, which already supports algebraic subtyping, with type classes.
In order to ensure type class coherence we argue for a restrictive way of implementing instances for types that are in the subtyping relation.

% \newpage
% \section*{Acknowledgements}

% Write here your acknowledgements.

\cleardoublepage

%%%%%%%%%%%%%%%%%%%%%%%%%%%%%%%%%%%%%%%%%%%%%%%%%%%%%%%%%%%%%%%%%%%%%%%%%%%%%
%%% Table of Contents
%%%%%%%%%%%%%%%%%%%%%%%%%%%%%%%%%%%%%%%%%%%%%%%%%%%%%%%%%%%%%%%%%%%%%%%%%%%%%

\renewcommand{\baselinestretch}{1.3}
\small\normalsize

\tableofcontents

\renewcommand{\baselinestretch}{1}
\small\normalsize

\cleardoublepage

%%%%%%%%%%%%%%%%%%%%%%%%%%%%%%%%%%%%%%%%%%%%%%%%%%%%%%%%%%%%%%%%%%%%%%%%%%%%%
%%% Der Haupttext, ab hier mit arabischer Numerierung
%%% Mit \input{dateiname} werden die Datei `dateiname' eingebunden
%%%%%%%%%%%%%%%%%%%%%%%%%%%%%%%%%%%%%%%%%%%%%%%%%%%%%%%%%%%%%%%%%%%%%%%%%%%%%

\pagenumbering{arabic}
\setcounter{page}{1}

%% Introduction
%%%%%%%%%%%%%%%%%%%%%%%%%%%%%%%%%%%%%%%%%%%%%%%%%%%%%%%%%%%%%%%%%%%%
% Introduction
%%%%%%%%%%%%%%%%%%%%%%%%%%%%%%%%%%%%%%%%%%%%%%%%%%%%%%%%%%%%%%%%%%%%

\chapter{Introduction}\label{ch:intro}

% this section likely needs a big overhaul
In many functional programming languages such as Haskell, type classes are a powerful tool to generalize functions over different data types.
This allows us e.g. to use the \mintinline{Haskell}|+| operator both on \mintinline{Haskell}|Int| and \mintinline{Haskell}|Float| types.
Another approach to overloading functions is subtyping, i.e. if a value of a certain type is expected we can also supply a value of a more specific type that is subsumed by the general type.
For example, since the type of natural numbers \mintinline{text}|Nat| is a subtype of the integers \mintinline{text}|Int|, we can supply a value of type \mintinline{text}|Nat| for any function that expects an \mintinline{text}|Int|.

Although these approaches do not serve exactly the same purpose it is uncommon to find both concepts in the same language.
In this work, I am going to show how it is possible to implement type classes in a language that supports subtyping.
There are unique challenges when bringing both together because instances for certain types are going to be ambiguous.

The Haskell type \mintinline{Haskell}|Either a b| roughly corresponds to the lattice type \mintinline{text}|a \/ b|.
Given a type class \mintinline{Haskell}|C :: * -> *| and instances \mintinline{Haskell}|C a| and \mintinline{Haskell}|C b| the instance \mintinline{Haskell}|C (Either a b)| has to be defined by hand which can be easily done by pattern-matching and using the given instances.
However, instances for lattice types are neither explicit nor straightforward:
Since \mintinline{text}|a \/ b| does not have a uniquely-determined(?) constructor we might just implicitly derive \mintinline{text}|C (a \/ b)| from the given instances.
However, this would make instances undecidable if we later on decide to implement an explicit instance of \mintinline{text}|C (a \/ b)|.

Together with subtyping we may be able to overload type classes even more.
Consider
\begin{minted}{text}
    Show(if b then 42 : Nat else "Hello World" : String)
\end{minted}
This term of type \mintinline{text}{Nat \/ String} appears to be well typed iff we can resolve the type class instances for \mintinline{text}{Show Nat} and \mintinline{text}{Show String}.

In the following we will explore how type classes interact with these and other lattice types.

\cleardoublepage

%% 
\chapter{Type Classes}\label{ch:typeclasses}

In order to generalize functions over data types there have been several proposals to abstract over types in different programming languages:
Amongst others,
\begin{itemize}
\item Parametric polymorphism: we can define functions without knowing the concrete representation of the arguments and result types (e.g. \texttt{id : a -> a}) % cite something
\item ad-hoc polymorphism: in some mostly imperative languages it is possible to simply overload functions (e.g. we may define \texttt{+} both on integers and on float values, the correct implementation is then picked based on the argument type) % cite...
\item and type classes. Combining both approaches (making ad-polymorphism less ad-hoc) we can define classes and corresponding functions and then use them to constrain valid argument types when defining a function that relies on these type class functions \cite{wadlerblott}
\end{itemize}

\section{Type Classes}\label{sec:typeclasses}
% overloading vs type classes
% paper by wadler

If our type system doesn't allow for ad-hoc polymorphism it may seem neccessary to write verbose code for basic function with respect to every concrete type it should be used for.
An intuitive example for this (that is also the motivation for type classes in the original proposal) are arithmetic operators.
We simply cannot define \texttt{(+) : Int -> Int -> Int} and then also \texttt{(+) : Float -> Float -> Float} in a different implementation, on which both types definetly rely on under the hood.
But these types have nonetheless something in common. Namely that they both stand for \emph{numerical} values, that hence support the usual arithmetic operations, like addition, multiplication, division and so on.

The idea of type classes is to generalise attributes of types with appropriate function.
The class \mintinline{Haskell}|Num| in Haskell expects that we can implement a number of numerical functions for a type $\tau$ if it is ought to be a member of the \mintinline{Haskell}|Num| type class.

Shortened definition of the \mintinline{Haskell}|Num| type class in Haskell.
\footnote{As can be found in the default \texttt{Prelude}: \url{https://hackage.haskell.org/package/base-4.16.2.0/docs/Prelude.html}}

\begin{minted}{Haskell}
class  Num a  where
    (+), (-), (*)       :: a -> a -> a
\end{minted}

An instance would look like:

\begin{minted}{Haskell}
instance  Num Int  where
    (+) = intAdd
    (-) = intMinus
    (*) = intMul
\end{minted}





% later:
% category theory: functors, monoids, monads, etc.

In the end, this enables us to use elaborate concepts such as functors and monads to reason about programs.

\cite{wadlerblott}

\section{Multi parameter type classes}\label{sec:multiparam}

\section{The problem of type class coherence}\label{sec:coherence}
% for each type there t may be globally at most one instance C t defined
% can already be tricky in a modular system
% more challenging with subtyping

Uniqueness or non-ambiguity of type classes.

\cleardoublepage

%% 

\chapter{Subtyping}\label{ch:subtyping}

\section{}
\cleardoublepage

%%
\chapter{Type Inference} \label{ch:inference}

Type checking of instance declarations:

\begin{mathpar}
  \inferrule{\mathcal{D}}{}
\end{mathpar}

\begin{lstlisting}
  instance Foo Bar { ... }
\end{lstlisting}

  \begin{minted}[escapeinside=??]{text}
    instance ?$\tau$? C {
       Method?$_1$?(?$x_1,...,x_n,k_1,...,k_n$?) => cmd
    };
  \end{minted}

Type checking of type class method calls:

  \begin{prooftree}
    \alwaysNoLine
    \AxiomC{$\mathcal{D}$}
    \alwaysSingleLine
    \UnaryInfC{$x:\sigma \to \tau \to \sigma, y: \tau\vdash \lambda z.xz(yz) : \sigma \to \sigma$}
      \introd
    \UnaryInfC{$x:\sigma \to \tau \to \sigma\vdash \lambda yz.xz(yz) : \tau \to \sigma \to \sigma$}
      \introd
    \UnaryInfC{$\vdash \lambda xyz.xz(yz) : (\sigma \to \tau \to \sigma) \to (\tau \to \sigma \to \sigma)$}
    \AxiomC{$x:\sigma,y:\tau\vdash x : \sigma$}
      \introd
    \UnaryInfC{$x:\sigma\vdash \lambda y.x : \tau \to \sigma$}
      \introd
    \UnaryInfC{$\vdash \lambda xy.x : \sigma \to \tau \to \sigma$}
      \elim
    \BinaryInfC{$\vdash (\lambda xyz.xz(yz)) (\lambda xy.x) : \tau \to \sigma \to \sigma$}
  \end{prooftree}

  % Main thesis for class coherence with subtyping
  Given \texttt{instance C a} and $sub < a$ and $a < sup$, we can neither have \texttt{instance C sub}, nor \texttt{instance C sup}.

  Could we just use the most specific instance? This might have unexpected results.
  E.g. if we have \texttt{NonEmptyList < List}, we may not know during compilation whether \texttt{NonEmptyList} or \texttt{List} is being picked.
  ~Generally to infer the mose specific type seems very hard. In this example filtering a \texttt{NonEmptyList} may or may not return an empty list and we may just have to assume that it is possibly empty.
  This may lead to hard to track behaviour when using overlapping instances.

  We should always check in an instance declaration whether this constraint globally holds. \\
  To guarantee modularity we also have to check this for module imports (possibly hiding instances).

  Discuss instance chains:
  We can relax this constraint by defining an explicit order in which instances should be picked/resolved.
  \cite{morris2010instance}
\cleardoublepage

%%
%%%%%%%%%%%%%%%%%%%%%%%%%%%%%%%%%%%%%%%%%%%%%%%%%%%%%%%%%%%%%%%%%%%%
% Diskussion und Ausblick
%%%%%%%%%%%%%%%%%%%%%%%%%%%%%%%%%%%%%%%%%%%%%%%%%%%%%%%%%%%%%%%%%%%%

\chapter{Summary}
  \label{ch:summary}


\clearpage

\cleardoublepage

%%
%%%%%%%%%%%%%%%%%%%%%%%%%%%%%%%%%%%%%%%%%%%%%%%%%%%%%%%%%%%%%%%%%%%%
% Diskussion und Ausblick
%%%%%%%%%%%%%%%%%%%%%%%%%%%%%%%%%%%%%%%%%%%%%%%%%%%%%%%%%%%%%%%%%%%%

\chapter{Discussion and Outlook}
\label{ch:discussion}

\section{Future Work}

\subsection{Multi Parameter Type Classes}

The concrete implementation may impose further restrictions on instance resolution.

Our language differentiattes covariant and contravariant type classes.
This distinction is made for the type variable declared in the class declaration.

In a covariant type class, type variables may only occur on covariant positions in the class methods signatures.
A simple example for a covariant type class is the \mintinline{Haskell}{Show} class:

\begin{minted}{Haskell}
  class Show +a where
    show :: a -> String
\end{minted}

Dually in a contravariant type class, type variables may only occur on contravariant positions in the class methods signatures.
A simple example for a contravariant type class is the \mintinline{Haskell}{Read} class:

\begin{minted}{Haskell}
  class Read -a where
    read :: Read -> a
\end{minted}

This distinction imposes a problem when we want to implement type classes in which the type variable may occur both in a covariant and contravariant position.

\begin{minted}{Haskell}
  class Semigroup a where
    mappend :: a -> a -> a
\end{minted}

One way to solve this problem is by distinguishing covariant and contravariant type variables.
We can do so by introducing multi parameter type classes:

\begin{minted}{Haskell}
  class Semigroup +a -b where
    mappend :: a -> a -> b
\end{minted}

Instead of defining an instance for \mintinline{Haskell}{Semigroup Nat}, we would then have to define an instance for \mintinline{Haskell}{Semigroup Nat Nat}.
The latter seems to be less intuitive because semigroups are not a relation between types but a property for just one type (the operator has to map two elements from one set into the same set).
Since such cases may occur in many other classes (e.g. Num, Monad) it may be helpful to define syntactic sugar for type classes with mixed variance.
Then, the less intuitive implementation as multi parameter type classes could be hidden on the surface syntax.

Discuss instance chains:
We can relax this constraint by defining an explicit order in which instances should be picked/resolved.
\cite{morris2010instance}

\cleardoublepage


%%%%%%%%%%%%%%%%%%%%%%%%%%%%%%%%%%%%%%%%%%%%%%%%%%%%%%%%%%%%%%%%%%%%%%%%%%%%%
%%% Appendix
%%%%%%%%%%%%%%%%%%%%%%%%%%%%%%%%%%%%%%%%%%%%%%%%%%%%%%%%%%%%%%%%%%%%%%%%%%%%%
\appendix



\cleardoublepage

%%%%%%%%%%%%%%%%%%%%%%%%%%%%%%%%%%%%%%%%%%%%%%%%%%%%%%%%%%%%%%%%%%%%%%%%%%%%%
%%% Bibliographie
%%%%%%%%%%%%%%%%%%%%%%%%%%%%%%%%%%%%%%%%%%%%%%%%%%%%%%%%%%%%%%%%%%%%%%%%%%%%%

\addcontentsline{toc}{chapter}{Bibliography}

\bibliographystyle{alpha}
\bibliography{literature}

\cleardoublepage
%%%%%%%%%%%%%%%%%%%%%%%%%%%%%%%%%%%%%%%%%%%%%%%%%%%%%%%%%%%%%%%%%%%%%%%%%%%%%
%%% Erklaerung
%%%%%%%%%%%%%%%%%%%%%%%%%%%%%%%%%%%%%%%%%%%%%%%%%%%%%%%%%%%%%%%%%%%%%%%%%%%%%
\thispagestyle{empty}
\section*{Selbst\"andigkeitserkl\"arung}

Hiermit versichere ich, dass ich die vorliegende Masterarbeit 
selbst\"andig und nur mit den angegebenen Hilfsmitteln angefertigt habe und dass alle Stellen, die dem Wortlaut oder dem 
Sinne nach anderen Werken entnommen sind, durch Angaben von Quellen als 
Entlehnung kenntlich gemacht worden sind. 
Diese Masterarbeit wurde in gleicher oder \"ahnlicher Form in keinem anderen 
Studiengang als Pr\"ufungsleistung vorgelegt. 

\vskip 3cm

Ort, Datum	\hfill Unterschrift \hfill 
%%%%%%%%%%%%%%%%%%%%%%%%%%%%%%%%%%%%%%%%%%%%%%%%%%%%%%%%%%%%%%%%%%%%%%%%%%%%%
%%% Ende
%%%%%%%%%%%%%%%%%%%%%%%%%%%%%%%%%%%%%%%%%%%%%%%%%%%%%%%%%%%%%%%%%%%%%%%%%%%%%

\end{document}

