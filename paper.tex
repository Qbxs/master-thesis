%%%%%%%%%%%%%%%%%%%%%%%%%%%%%%%%%%%%%%%%%%%%%%%%%%%%%%%%%%%%%%%%%%%%%%%%%%%%%
%%% LaTeX-Rahmen fuer das Erstellen von Masterarbeiten
%%%%%%%%%%%%%%%%%%%%%%%%%%%%%%%%%%%%%%%%%%%%%%%%%%%%%%%%%%%%%%%%%%%%%%%%%%%%%

%%%%%%%%%%%%%%%%%%%%%%%%%%%%%%%%%%%%%%%%%%%%%%%%%%%%%%%%%%%%%%%%%%%%%%%%%%%%%
%%% allgemeine Einstellungen
%%%%%%%%%%%%%%%%%%%%%%%%%%%%%%%%%%%%%%%%%%%%%%%%%%%%%%%%%%%%%%%%%%%%%%%%%%%%%

\documentclass[twoside,12pt,a4paper]{report}
%\usepackage{reportpage}
\usepackage{amsmath}
\usepackage{scalerel}
\usepackage{epsf}
\usepackage{graphics, graphicx}
\usepackage{latexsym}
\usepackage[margin=10pt,font=small,labelfont=bf]{caption}
\usepackage[utf8]{inputenc}
\usepackage[toc,page]{appendix}
\usepackage{tikz-cd}
\usetikzlibrary{cd}

\usepackage{amsthm}
\newtheorem{definition}{Definition}

\usepackage[hyphens]{url} % 'hyphens' option allows line breaks after "-" characters
\usepackage[ colorlinks = true
           , linkcolor = black
           , anchorcolor = black
           , citecolor = black
           , urlcolor = blue]{hyperref}

\usepackage{listings}
% basic duo style
\lstdefinestyle{duoStyle}{%
    morekeywords={if, then, else instance, class, >>, CBV, CBN, +, -, =>, case, of, cocase, mu, def, prd, cns, cmd, :=},
    keywordstyle=\ttfamily\bfseries\color{blue!80!black},
    %numbers=none, % do not uncomment! issues when used inside arrays!
    mathescape=true
    %moredelim=**[is][\bfseries]{|}{|}
}
\lstset{style=duoStyle}

\usepackage{minted}


\usepackage{bussproofs}
\EnableBpAbbreviations
\def\ScoreOverhang{0pt}
\def\buildNoScoreTight{% Make an hbox with no score
 \global\setbox\myBoxD=\hbox{\vbox{\vskip0pt}}% 0pt instead of 1pt
}
\def\noLine{% noLine with less space between lines
 \gdef\buildScore{\buildNoScoreTight}%
 \ignorespaces%
}

% Inference rules
\newcommand{\subPosRule}{\RightLabel{\textsc{Sub}$^+$}}
\newcommand{\subNegRule}{\RightLabel{\textsc{Sub}$^-$}}
\newcommand{\instanceDeclRule}{\RightLabel{\textsc{Decl}}}
\newcommand{\meetRule}{\RightLabel{\textsc{Meet}}}
\newcommand{\joinRule}{\RightLabel{\textsc{Join}}}
\newcommand{\axiomPos}{\RightLabel{\textsc{Axiom}$^+$}}
\newcommand{\axiomNeg}{\RightLabel{\textsc{Axiom}$^-$}}
\newcommand{\topRule}{\RightLabel{\textsc{Top}}}
\newcommand{\botRule}{\RightLabel{\textsc{Bot}}}

\newcommand{\ctx}{\ensuremath{\Gamma \; \vdash \;}}

% Symbols
\newcommand\join{\scalerel*{\bigvee}{i}}
\newcommand\meet{\scalerel*{\bigwedge}{i}}
\newcommand\sub{\ensuremath{:<}}

\usepackage{mathpartir}

\textwidth 14cm
\textheight 22cm
\topmargin 0.0cm
\evensidemargin 1cm
\oddsidemargin 1cm
%\footskip 2cm
\parskip0.5explus0.1exminus0.1ex

% Kann von Student auch nach pers\"onlichem Geschmack ver\"andert werden.
\pagestyle{headings}

\sloppy

% not working as intended
% \newcommand{\textHaskell}{1}{\mint{Haskell}|$1|}

\begin{document}

%%%%%%%%%%%%%%%%%%%%%%%%%%%%%%%%%%%%%%%%%%%%%%%%%%%%%%%%%%%%%%%%%%%%%%%%%%%%
%%% hier steht die neue Titelseite 
%%%%%%%%%%%%%%%%%%%%%%%%%%%%%%%%%%%%%%%%%%%%%%%%%%%%%%%%%%%%%%%%%%%%%%%%%%%%
 
\begin{titlepage}
 \begin{center}
  {\LARGE Eberhard Karls Universit\"at T\"ubingen}\\
  {\large Mathematisch-Naturwissenschaftliche Fakult\"at \\
Wilhelm-Schickard-Institut f\"ur Informatik\\[4cm]}
  {\huge Master Thesis Informatics\\[2cm]}
  {\Large\bf  Type Class Coherence with Subtyping\\[1.5cm]}
 {\large Pascal Engel}\\[0.5cm]
Datum\\[4cm]
{\small\bf Reviewers}\\[0.5cm]
  \parbox{7cm}{\begin{center}{\large Name Erstgutachter}\\
   (Informatik)\\
  {\footnotesize Wilhelm-Schickard-Institut f\"ur Informatik\\
	Universit\"at T\"ubingen}\end{center}}\hfill\parbox{7cm}{\begin{center}
  {\large Name Zweitgutachter}\\
  (Informatik)\\
  {\footnotesize Wilhelm-Schickard-Institut f\"ur Informatik\\
	Universit\"at T\"ubingen}\end{center}
 }
  \end{center}
\end{titlepage}

%%%%%%%%%%%%%%%%%%%%%%%%%%%%%%%%%%%%%%%%%%%%%%%%%%%%%%%%%%%%%%%%%%%%%%%%%%%%
%%% Titelr"uckseite: Bibliographische Angaben
%%%%%%%%%%%%%%%%%%%%%%%%%%%%%%%%%%%%%%%%%%%%%%%%%%%%%%%%%%%%%%%%%%%%%%%%%%%%

\thispagestyle{empty}
\vspace*{\fill}
\begin{minipage}{11.2cm}
\textbf{Engel, Pascal:}\\
\emph{Type Class Coherence with Subtyping}\\ Master Thesis Informatics\\
Eberhard Karls Universit\"at T\"ubingen\\
Thesis period: von-bis
\end{minipage}
\newpage

%%%%%%%%%%%%%%%%%%%%%%%%%%%%%%%%%%%%%%%%%%%%%%%%%%%%%%%%%%%%%%%%%%%%%%%%%%%%

\pagenumbering{roman}
\setcounter{page}{1}

%%%%%%%%%%%%%%%%%%%%%%%%%%%%%%%%%%%%%%%%%%%%%%%%%%%%%%%%%%%%%%%%%%%%%%%%%%%%
%%% Seite I: Zusammenfassug, Danksagung
%%%%%%%%%%%%%%%%%%%%%%%%%%%%%%%%%%%%%%%%%%%%%%%%%%%%%%%%%%%%%%%%%%%%%%%%%%%%


\section*{Abstract}

In this work we aim to bring together three notions of polymorphism:
Subtyping as a form of ad hoc polymorphism, parametric polymorphism and type classes.
We do so by extending the research language \texttt{duo}, which already supports algebraic subtyping, with type classes.
In order to ensure type class coherence we argue for a restrictive way of implementing instances for types that are in the subtyping relation.

% \newpage
% \section*{Acknowledgements}

% Write here your acknowledgements.

\cleardoublepage

%%%%%%%%%%%%%%%%%%%%%%%%%%%%%%%%%%%%%%%%%%%%%%%%%%%%%%%%%%%%%%%%%%%%%%%%%%%%%
%%% Table of Contents
%%%%%%%%%%%%%%%%%%%%%%%%%%%%%%%%%%%%%%%%%%%%%%%%%%%%%%%%%%%%%%%%%%%%%%%%%%%%%

\renewcommand{\baselinestretch}{1.3}
\small\normalsize

\tableofcontents

\renewcommand{\baselinestretch}{1}
\small\normalsize

\cleardoublepage

%%%%%%%%%%%%%%%%%%%%%%%%%%%%%%%%%%%%%%%%%%%%%%%%%%%%%%%%%%%%%%%%%%%%%%%%%%%%%
%%% Der Haupttext, ab hier mit arabischer Numerierung
%%% Mit \input{dateiname} werden die Datei `dateiname' eingebunden
%%%%%%%%%%%%%%%%%%%%%%%%%%%%%%%%%%%%%%%%%%%%%%%%%%%%%%%%%%%%%%%%%%%%%%%%%%%%%

\pagenumbering{arabic}
\setcounter{page}{1}

%% Introduction
%%%%%%%%%%%%%%%%%%%%%%%%%%%%%%%%%%%%%%%%%%%%%%%%%%%%%%%%%%%%%%%%%%%%
% Introduction
%%%%%%%%%%%%%%%%%%%%%%%%%%%%%%%%%%%%%%%%%%%%%%%%%%%%%%%%%%%%%%%%%%%%

\chapter{Introduction}\label{ch:intro}

% this section likely needs a big overhaul
In many functional programming languages such as Haskell, type classes are a powerful tool to generalize functions over different data types.
This allows us e.g. to use the \mintinline{Haskell}|+| operator both on \mintinline{Haskell}|Int| and \mintinline{Haskell}|Float| types.
Another approach to overloading functions is subtyping, i.e. if a value of a certain type is expected we can also supply a value of a more specific type that is subsumed by the general type.
For example, since the type of natural numbers \mintinline{text}|Nat| is a subtype of the integers \mintinline{text}|Int|, we can supply a value of type \mintinline{text}|Nat| for any function that expects an \mintinline{text}|Int|.

Although these approaches do not serve exactly the same purpose it is uncommon to find both concepts in the same language.
In this work, I am going to show how it is possible to implement type classes in a language that supports subtyping.
There are unique challenges when bringing both together because instances for certain types are going to be ambiguous.

The Haskell type \mintinline{Haskell}|Either a b| roughly corresponds to the lattice type \mintinline{text}|a \/ b|.
Given a type class \mintinline{Haskell}|C :: * -> *| and instances \mintinline{Haskell}|C a| and \mintinline{Haskell}|C b| the instance \mintinline{Haskell}|C (Either a b)| has to be defined by hand which can be easily done by pattern-matching and using the given instances.
However, instances for lattice types are neither explicit nor straightforward:
Since \mintinline{text}|a \/ b| does not have a uniquely-determined(?) constructor we might just implicitly derive \mintinline{text}|C (a \/ b)| from the given instances.
However, this would make instances undecidable if we later on decide to implement an explicit instance of \mintinline{text}|C (a \/ b)|.

Together with subtyping we may be able to overload type classes even more.
Consider
\begin{minted}{text}
    Show(if b then 42 : Nat else "Hello World" : String)
\end{minted}
This term of type \mintinline{text}{Nat \/ String} appears to be well typed iff we can resolve the type class instances for \mintinline{text}{Show Nat} and \mintinline{text}{Show String}.

In the following we will explore how type classes interact with these and other lattice types.

\cleardoublepage

%%
\chapter{Polymorphism}\label{ch:polymorphism}

In order to generalize functions over data types there have been several proposals to abstract over types in different programming languages:
These can be summarised in three categories: parametric polymorphism, subtyping and ad-hoc polymorphism.

\section{Parametric polymorphism}\label{sec:parmetric-polymorphism}

We can define functions without knowing the concrete representation of the arguments and result types (e.g. \texttt{id : a -> a}) % cite something
Abstract algorithms detached from concrete type representation.
Allows for 'theorems for free'. \cite{wadlertheorems}

\section{Subtyping}\label{sec:subtyping}

In many cases the specific semantics of types exhibit a hierarchy.
In Object-oriented programming this hierarchy is given in the form of sub- and superclasses.
We can express the relationship between super- and subclasses, or more generally super- and subtypes,
in the form that all properties of the superclass is also exhibited in the subclass. \cite{subtyping}

In the case of OO-languages this means that, if the class \texttt{SubC} is a subclass of \texttt{SuperC},
then any method defined in \texttt{SuperC} is also going to be defined for objects of \texttt{SubC}.
This enables us to use an object \texttt{SubC} wherever a \texttt{SuperC} is expected.

We denote $T \leq S$ for $T$ is a subtype of $S$.
Syntactically, this implies that if we have obatined the judgement $e : T$, we also have $e : S$.
Therefore, we can use $e$ in any context that expects the usage of a term of type $S$.
Semantically, the subtyping relation can be understood analogously to sets in terms of the subset relationship $\subseteq$,
meaning all terms $e$ of type $T$ are also of type $S$.
\cite{reynolds_1998}

This permits many useful features in programming languages such as the reuse and abstraction of code to the supertypes and implicit coercions from a subtype to a supertype.


\section{Ad-hoc polymorphism}\label{sec:ad-hoc-polymorphism}
% overloading vs type classes
% paper by wadler

In some - mostly imperative - languages it is possible to simply overload functions (e.g. we may define \texttt{+} both on integers and on float values, the correct implementation is then picked based on the argument type).
However, there is usually no way to express this in the type system of these languages. % cite...

If our type system doesn't allow for ad-hoc polymorphism it may seem neccessary to write verbose code for basic function with respect to every concrete type it should be used for.
An intuitive example for this (that is also the motivation for type classes in the original proposal) are arithmetic operators.
We simply cannot define \texttt{(+) : Int -> Int -> Int} and then also \texttt{(+) : Float -> Float -> Float} in a different implementation, on which both types definetly rely on under the hood.
But these types have nonetheless something in common. Namely that they both stand for \emph{numerical} values, that hence support the usual arithmetic operations, like addition, multiplication, division and so on.

The idea of type classes is to generalise attributes of types with appropriate function.
The class \mintinline{Haskell}|Num| in Haskell expects that we can implement a number of numerical functions for a type $\tau$ if it is ought to be a member of the \mintinline{Haskell}|Num| type class.

Shortened definition of the \mintinline{Haskell}|Num| type class in Haskell.
\footnote{As can be found in the default \texttt{Prelude}: \url{https://hackage.haskell.org/package/base-4.16.2.0/docs/Prelude.html\#t:Num}}

\begin{minted}{Haskell}
class  Num a  where
    (+), (-), (*) :: a -> a -> a
\end{minted}

An instance would look like:

\begin{minted}{Haskell}
instance  Num Int  where
    (+) = intAdd
    (-) = intMinus
    (*) = intMul
\end{minted}





% later:
% category theory: functors, monoids, monads, etc.

In the end, this enables us to use elaborate concepts such as functors and monads to reason about programs.

\cite{wadlerblott}

\section{The problem of type class coherence}\label{sec:coherence}
% for each type there t may be globally at most one instance C t defined
% discuss overlapping instances
% can already be tricky in a modular system
% more challenging with subtyping

Even though the general concept of type classes introduces a general meaning for each type class.
The evaluation still strongly depends on implementation details found in specific instances.
For example, for the \texttt{Ord} type class we may choose to implement the ordering in ascending or descending order.
It is therefore crucial, that for each type the corresponding instance - if it exists - is uniquely determined by the type.

Reynolds \cite{reynolds_coherence} describes the issue of coherence as follows:

\begin{quote}
    When a programming language has a sufficiently rich type structure, there can be more than one proof of the same
    typing judgment; potentially this can lead to semantic ambiguity since the semantics of a typed language is a function
    of such proofs. When no such ambiguity arises, we say that the language is coherent.
\end{quote}

For type classes, this means that no two instances should be able to be resolved for the same type.
A rather obvious example would be to define two different instances for the same type.
For example one instance of \mintinline{Haskell}{Ord Int}, one with ascending and one with descending order.
The ambiguity arises as soon as we make use of these instances and it is no longer clear which one should be picked for evaluation.

More surprisingly type class coherence is already violated for overlapping instances.
As part of the standard prelude we find both \mintinline{Haskell}{instance Show a => Show [a]} and \mintinline{Haskell}{instance Show String}
(with \mintinline{Haskell}{String} being a type synonym for \mintinline{Haskell}{[Char]}).
The \texttt{Show} instance for Strings differs from the more general instance for lists.

Ambiguous programs should generally not be typeable.
One example, also mentioned in the Haskell 98 report \cite{Haskell98} is this short program which simply reads a string to a data type and then converts it back to string without specifying which data type is being used:

\begin{minted}{Haskell}
    f :: String -> String
    f str = let x = read str in show x
\end{minted}

There may be multiple types that satisfy the type class constraints.
The specific implementation of \mintinline{Haskell}{show :: forall a. (Show a) => a -> String} and \mintinline{Haskell}{read :: forall a.(Read a) -> String -> a} is therefore unknown.

In the Haskell98 standard, type class coherence is guaranteed by the syntactical equivalence of resolved instances.
For each Haskell type there may be at most one instance defined for each type class.


\subsection{Example}

Consider superclasses in Haskell.
E.g. the type class \mintinline{Haskell}{Eq} is a superclass of \mintinline{Haskell}{Ord}, written \mintinline{Haskell}{Eq a => Ord a}.
This means that for each type that we want to define an ordering for already needs to have equality defined on.

For example, given an instance for \mintinline{Haskell}{Ord (Maybe a)} we can then derive that an instance for \mintinline{Haskell}{Eq a} has to exist.
As shown in the diagram, we can take different paths to do so but type class coherence guarantees that no matter which path we choose to resolve the instance, we will always find \emph{the same} instance.
In Haskell98 type class coherence guarantees that all such diagrams commute.
So even if there may be different ways to resolve type class constraints, all of them preserve the same semantics.

\begin{tikzcd}
    &  & \mintinline{Haskell}|Ord (Maybe a)| \arrow[llddd, "{\footnotesize\mintinline{Haskell}|class Eq a => Ord a|}"'] \arrow[rrddd, "{\footnotesize\mintinline{Haskell}|instance Ord a => Ord (Maybe a)|}"] &  &                            \\
    &  &                                &  &                            \\
    &  &                                &  &                            \\
    \mintinline{Haskell}|Eq (Maybe a)| \arrow[rrddd, "{\footnotesize\mintinline{Haskell}|instance Eq a => Eq (Maybe a)|}"'] &  &                                &  & \mintinline{Haskell}|Ord a| \arrow[llddd, "{\footnotesize \mintinline{Haskell}|class Eq a => Ord a|}"] \\
    &  &                                &  &                            \\
    &  &                                &  &                            \\
    &  & \mintinline{Haskell}|Eq a|                  &  &                           
\end{tikzcd}

Even though there are multiple ways to derive an instance for \mintinline{Haskell}|Eq a| from an instance of \texttt{Ord (Maybe a)}, the derived instance has to be uniquely determined.
Since the diagram commutes, there has to be exactly one instance for \mintinline{Haskell}|Eq a|.

Uniqueness or non-ambiguity of type classes.

\cleardoublepage

%% 
\chapter{Type Inference} \label{ch:inference}

Type checking of instance declarations:

\begin{mathpar}
  \inferrule{\mathcal{D}}{}
\end{mathpar}

\begin{lstlisting}
  instance Foo Bar { ... }
\end{lstlisting}

  \begin{minted}[escapeinside=??]{text}
    instance ?$\tau$? C {
       Method?$_1$?(?$x_1,...,x_n,k_1,...,k_n$?) => cmd
    };
  \end{minted}

Type checking of type class method calls:

  \begin{prooftree}
    \alwaysNoLine
    \AxiomC{$\mathcal{D}$}
    \alwaysSingleLine
    \UnaryInfC{$x:\sigma \to \tau \to \sigma, y: \tau\vdash \lambda z.xz(yz) : \sigma \to \sigma$}
      \introd
    \UnaryInfC{$x:\sigma \to \tau \to \sigma\vdash \lambda yz.xz(yz) : \tau \to \sigma \to \sigma$}
      \introd
    \UnaryInfC{$\vdash \lambda xyz.xz(yz) : (\sigma \to \tau \to \sigma) \to (\tau \to \sigma \to \sigma)$}
    \AxiomC{$x:\sigma,y:\tau\vdash x : \sigma$}
      \introd
    \UnaryInfC{$x:\sigma\vdash \lambda y.x : \tau \to \sigma$}
      \introd
    \UnaryInfC{$\vdash \lambda xy.x : \sigma \to \tau \to \sigma$}
      \elim
    \BinaryInfC{$\vdash (\lambda xyz.xz(yz)) (\lambda xy.x) : \tau \to \sigma \to \sigma$}
  \end{prooftree}

  % Main thesis for class coherence with subtyping
  Given \texttt{instance C a} and $sub < a$ and $a < sup$, we can neither have \texttt{instance C sub}, nor \texttt{instance C sup}.

  Could we just use the most specific instance? This might have unexpected results.
  E.g. if we have \texttt{NonEmptyList < List}, we may not know during compilation whether \texttt{NonEmptyList} or \texttt{List} is being picked.
  ~Generally to infer the mose specific type seems very hard. In this example filtering a \texttt{NonEmptyList} may or may not return an empty list and we may just have to assume that it is possibly empty.
  This may lead to hard to track behaviour when using overlapping instances.

  We should always check in an instance declaration whether this constraint globally holds. \\
  To guarantee modularity we also have to check this for module imports (possibly hiding instances).

  Discuss instance chains:
  We can relax this constraint by defining an explicit order in which instances should be picked/resolved.
  \cite{morris2010instance}
\cleardoublepage

%%
%%%%%%%%%%%%%%%%%%%%%%%%%%%%%%%%%%%%%%%%%%%%%%%%%%%%%%%%%%%%%%%%%%%%
% Diskussion und Ausblick
%%%%%%%%%%%%%%%%%%%%%%%%%%%%%%%%%%%%%%%%%%%%%%%%%%%%%%%%%%%%%%%%%%%%

\chapter{Summary}
  \label{ch:summary}


\clearpage

\cleardoublepage

%%
%%%%%%%%%%%%%%%%%%%%%%%%%%%%%%%%%%%%%%%%%%%%%%%%%%%%%%%%%%%%%%%%%%%%
% Diskussion und Ausblick
%%%%%%%%%%%%%%%%%%%%%%%%%%%%%%%%%%%%%%%%%%%%%%%%%%%%%%%%%%%%%%%%%%%%

\chapter{Discussion and Outlook}
\label{ch:discussion}

\section{Future Work}

\subsection{Multi Parameter Type Classes}

The concrete implementation may impose further restrictions on instance resolution.

Our language differentiattes covariant and contravariant type classes.
This distinction is made for the type variable declared in the class declaration.

In a covariant type class, type variables may only occur on covariant positions in the class methods signatures.
A simple example for a covariant type class is the \mintinline{Haskell}{Show} class:

\begin{minted}{Haskell}
  class Show +a where
    show :: a -> String
\end{minted}

Dually in a contravariant type class, type variables may only occur on contravariant positions in the class methods signatures.
A simple example for a contravariant type class is the \mintinline{Haskell}{Read} class:

\begin{minted}{Haskell}
  class Read -a where
    read :: Read -> a
\end{minted}

This distinction imposes a problem when we want to implement type classes in which the type variable may occur both in a covariant and contravariant position.

\begin{minted}{Haskell}
  class Semigroup a where
    mappend :: a -> a -> a
\end{minted}

One way to solve this problem is by distinguishing covariant and contravariant type variables.
We can do so by introducing multi parameter type classes:

\begin{minted}{Haskell}
  class Semigroup +a -b where
    mappend :: a -> a -> b
\end{minted}

Instead of defining an instance for \mintinline{Haskell}{Semigroup Nat}, we would then have to define an instance for \mintinline{Haskell}{Semigroup Nat Nat}.
The latter seems to be less intuitive because semigroups are not a relation between types but a property for just one type (the operator has to map two elements from one set into the same set).
Since such cases may occur in many other classes (e.g. Num, Monad) it may be helpful to define syntactic sugar for type classes with mixed variance.
Then, the less intuitive implementation as multi parameter type classes could be hidden on the surface syntax.

Discuss instance chains:
We can relax this constraint by defining an explicit order in which instances should be picked/resolved.
\cite{morris2010instance}

\cleardoublepage


%%%%%%%%%%%%%%%%%%%%%%%%%%%%%%%%%%%%%%%%%%%%%%%%%%%%%%%%%%%%%%%%%%%%%%%%%%%%%
%%% Appendix
%%%%%%%%%%%%%%%%%%%%%%%%%%%%%%%%%%%%%%%%%%%%%%%%%%%%%%%%%%%%%%%%%%%%%%%%%%%%%
\appendix



\cleardoublepage

%%%%%%%%%%%%%%%%%%%%%%%%%%%%%%%%%%%%%%%%%%%%%%%%%%%%%%%%%%%%%%%%%%%%%%%%%%%%%
%%% Bibliographie
%%%%%%%%%%%%%%%%%%%%%%%%%%%%%%%%%%%%%%%%%%%%%%%%%%%%%%%%%%%%%%%%%%%%%%%%%%%%%

\addcontentsline{toc}{chapter}{Bibliography}

\bibliographystyle{alpha}
\bibliography{literature}

\cleardoublepage
%%%%%%%%%%%%%%%%%%%%%%%%%%%%%%%%%%%%%%%%%%%%%%%%%%%%%%%%%%%%%%%%%%%%%%%%%%%%%
%%% Erklaerung
%%%%%%%%%%%%%%%%%%%%%%%%%%%%%%%%%%%%%%%%%%%%%%%%%%%%%%%%%%%%%%%%%%%%%%%%%%%%%
\thispagestyle{empty}
\section*{Selbst\"andigkeitserkl\"arung}

Hiermit versichere ich, dass ich die vorliegende Masterarbeit 
selbst\"andig und nur mit den angegebenen Hilfsmitteln angefertigt habe und dass alle Stellen, die dem Wortlaut oder dem 
Sinne nach anderen Werken entnommen sind, durch Angaben von Quellen als 
Entlehnung kenntlich gemacht worden sind. 
Diese Masterarbeit wurde in gleicher oder \"ahnlicher Form in keinem anderen 
Studiengang als Pr\"ufungsleistung vorgelegt. 

\vskip 3cm

Ort, Datum	\hfill Unterschrift \hfill 
%%%%%%%%%%%%%%%%%%%%%%%%%%%%%%%%%%%%%%%%%%%%%%%%%%%%%%%%%%%%%%%%%%%%%%%%%%%%%
%%% Ende
%%%%%%%%%%%%%%%%%%%%%%%%%%%%%%%%%%%%%%%%%%%%%%%%%%%%%%%%%%%%%%%%%%%%%%%%%%%%%

\end{document}

